\documentclass{ctexbook}
\usepackage{amsmath}
\usepackage{geometry}
\usepackage{amssymb}
\usepackage{graphicx}
\usepackage{booktabs}
\usepackage{extarrows}
\usepackage{multirow}
\usepackage{makecell}
\usepackage{hyperref}
\usepackage{amsmath}
\usepackage{amssymb}
\usepackage{pgfplots}
\usepackage{mathdots}
\usepackage{float}
\usepackage{amsthm}
\usepackage{bm}
\usepackage{esint}
\theoremstyle{remark}
\newtheorem{remark}{注}
\usepackage{amsmath, amssymb}
\usepackage{enumitem} % 用于调整列表格式
\usepackage{titlesec} % 用于自定义章节样式
\usepackage{lipsum}   % 用于生成占位文本(可删除)

% 自定义章节样式
\titleformat{\chapter}[display]
{\normalfont\bfseries\centering}
{\chaptertitlename\ \thechapter}{10pt}{\Large}

% 自定义小节样式
\titleformat{\section}[hang]
{\normalfont\bfseries}
{\thesection}{1em}{}

\begin{document}
	
	\chapter{计算机系统概述}
	
	\section*{【考纲内容】}
	
	\begin{enumerate}
		\item 操作系统的基本概念
		\item 操作系统的发展历程
		\item 程序运行环境
		
		
CPU 运行模式:内核模式与用户模式;中断和异常的处理;系统调用;
		
程序的链接与装入;程序运行时内存映像与地址空间\footnote{此处为脚注示例。}
		
		\item 操作系统结构
		
		
分层、模块化、宏内核、微内核、外核
		
		\item 操作系统引导
		\item 虚拟机
	\end{enumerate}
	
	\section*{【复习提示】}
	
	本章通常以选择题的形式考查,重点考查操作系统的功能、运行环境和提供的服务。要求读者能从宏观上把握操作系统各部分的功能,微观上掌握细微的知识点。因此,复习操作系统时,首先要形成大体框架,并通过反复复习和做题巩固知识体系,然后将操作系统的所有内容串成一个整体。本章的内容有助于读者整体上初步认识操作系统,为后面掌握各章节的知识点奠定基础,进而整体把握课程,不要因为本章的内容在历年考题中出现的比例不高而忽视它。
	
\end{document}