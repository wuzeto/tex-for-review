% Options for packages loaded elsewhere
\PassOptionsToPackage{unicode}{hyperref}
\PassOptionsToPackage{hyphens}{url}
\documentclass[UTF8]{ctexart}
\usepackage{xcolor}
\usepackage{amsmath,amssymb}
\setcounter{secnumdepth}{-\maxdimen} % remove section numbering
\usepackage{iftex}
\ifPDFTeX
  \usepackage[T1]{fontenc}
  \usepackage[utf8]{inputenc}
  \usepackage{textcomp} % provide euro and other symbols
\else % if luatex or xetex
  \usepackage{unicode-math} % this also loads fontspec
  \defaultfontfeatures{Scale=MatchLowercase}
  \defaultfontfeatures[\rmfamily]{Ligatures=TeX,Scale=1}
\fi
\usepackage{lmodern}
\ifPDFTeX\else
  % xetex/luatex font selection
\fi
% Use upquote if available, for straight quotes in verbatim environments
\IfFileExists{upquote.sty}{\usepackage{upquote}}{}
\IfFileExists{microtype.sty}{% use microtype if available
  \usepackage[]{microtype}
  \UseMicrotypeSet[protrusion]{basicmath} % disable protrusion for tt fonts
}{}
\makeatletter
\@ifundefined{KOMAClassName}{% if non-KOMA class
  \IfFileExists{parskip.sty}{%
    \usepackage{parskip}
  }{% else
    \setlength{\parindent}{0pt}
    \setlength{\parskip}{6pt plus 2pt minus 1pt}}
}{% if KOMA class
  \KOMAoptions{parskip=half}}
\makeatother
\setlength{\emergencystretch}{3em} % prevent overfull lines
\providecommand{\tightlist}{%
  \setlength{\itemsep}{0pt}\setlength{\parskip}{0pt}}
\usepackage{bookmark}
\IfFileExists{xurl.sty}{\usepackage{xurl}}{} % add URL line breaks if available
\urlstyle{same}
\hypersetup{
  hidelinks,
  pdfcreator={LaTeX via pandoc}}
\usepackage[a4paper, margin=0.6in]{geometry}
\author{}
\date{}

\begin{document}

\section{2025年高考数学超难综合模拟卷}\label{2025ux5e74ux9ad8ux8003ux6570ux5b66ux8d85ux96beux7efcux5408ux6a21ux62dfux5377}

\subsection{一、单项选择题(每题5分,共40分)}\label{ux4e00ux5355ux9879ux9009ux62e9ux9898ux6bcfux98985ux5206ux517140ux5206uxff09}

\begin{enumerate}
\def\labelenumi{\arabic{enumi}.}
\item
  设函数
  \( f(x) = \int_{0}^{x^2} \frac{\arcsin\sqrt{t}}{1+t^3} dt \),则
  \( \lim_{x \to 0} \frac{f(x) - x^3 \int_{0}^{x} e^{-t^4} dt}{x^7 \ln(1+x^2)} = \)\\
  (A) \( -\frac{1}{210} \) (B) \( \frac{1}{210} \) (C)
  \( -\frac{1}{180} \) (D) \( \frac{1}{180} \)
\item
  已知7阶矩阵 \( A \) 满足 \( A^4 = O \) 且
  \( \text{rank}(A^3) = 1 \),则 \( A \)
  的Jordan标准形中2阶Jordan块的数量为\\
  (A) 2 (B) 3 (C) 4 (D) 5
\item
  设二维随机变量 \( (X,Y) \) 满足 \( X \sim \text{Exp}(1) \),且在
  \( X=x \) 条件下 \( Y \sim \text{Gamma}(2,x) \),则
  \( \text{Cov}(X,Y) \) 和 \( E(Y^3) \) 分别为\\
  (A) 1, 12 (B) 2, 24 (C) 1, 24 (D) 2, 48
\item
  微分方程 \( (x^2 y''')^2 + \sin(y') = x \log_2 x \)
  的奇解存在性判断及最高阶导数系数分析为\\
  (A) 存在奇解,最高阶导数系数为变系数\\
  (B) 不存在奇解,最高阶导数系数为常系数\\
  (C) 存在奇解,最高阶导数系数为常系数\\
  (D) 不存在奇解,最高阶导数系数为变系数
\item
  设 \( \Omega \) 由曲面 \( z = x^2 + y^2 - 1 \) 与
  \( z = 1 - x^2 - y^2 \) 围成,计算
  \( \iiint_{\Omega} (x^5 + y^5 + z^5) dV \)
  时,利用对称性化简后非零项为\\
  (A) \( 2 \iiint_{\Omega} z^5 dV \) (B)
  \( 4 \iiint_{\Omega} x^5 dV \)\\
  (C) \( 8 \iiint_{\Omega} y^5 dV \) (D) 所有项均为零
\item
  级数 \( \sum_{n=2}^{\infty} \frac{(-1)^n}{n(\ln n)^{\alpha}} \) 与
  \( \sum_{n=2}^{\infty} \frac{1}{n(\ln n)(\ln\ln n)^{\beta}} \)
  同时收敛的 \( (\alpha,\beta) \) 范围为\\
  (A) \( \alpha > 1, \beta > 1 \) (B) \( \alpha \geq 1, \beta > 1 \) \\
  (C) \( \alpha > 1, \beta \geq 1 \) (D)
  \( \alpha \geq 1, \beta \geq 1 \)
\item
  设向量组 \( \alpha_1, \alpha_2, \alpha_3, \alpha_4 \) 线性无关,向量组
  \( \beta_1 = \alpha_1 + k\alpha_4 \),
  \( \beta_2 = \alpha_2 + k\alpha_1 \),
  \( \beta_3 = \alpha_3 + k\alpha_2 \),
  \( \beta_4 = \alpha_4 + k\alpha_3 \),当 \( k^4 = \) 何值时
  \( \beta_1, \beta_2, \beta_3, \beta_4 \) 线性相关?\\
  (A) 1 (B) -1 (C) 2 (D) -2
\item
  设总体 \( X \sim \text{Pareto}(3,\theta) \),其概率密度为
  \( f(x;\theta) = \frac{3\theta^3}{x^4}, x \geq \theta \),\( X_1,X_2,X_3 \)
  为样本,下列估计量中满足渐近正态性的 \( \theta \) 估计量是\\
  (A) \( \hat{\theta}_1 = \frac{3}{4} \min\{X_1,X_2,X_3\} \) (B)
  \( \hat{\theta}_2 = \frac{1}{3} \sum_{i=1}^3 X_i \)\\
  (C) \( \hat{\theta}_3 = \sqrt[3]{\prod_{i=1}^3 X_i} \) (D)
  \( \hat{\theta}_4 = \frac{\max\{X_1,X_2,X_3\}}{2} \)
\end{enumerate}

\subsection{二、填空题(每题5分,共30分)}\label{ux4e8cux586bux7a7aux9898ux6bcfux98985ux5206ux517130ux5206uxff09}

\begin{enumerate}
\def\labelenumi{\arabic{enumi}.}
\item
  求极限
  \( \lim_{n \to \infty} n^3 \left(1 - \cos\frac{1}{\sqrt{n}} - \int_0^{\frac{1}{\sqrt{n}}} \frac{\arctan t}{1+t^2} dt\right) = \)
  \textbf{\_\_}
\item
  设矩阵 \( A = \begin{pmatrix} 
  1 & 1 & 1 \\
  1 & \omega & \omega^2 \\
  1 & \omega^2 & \omega 
  \end{pmatrix} \)(其中 \( \omega \) 为三次单位根),则二次型
  \( f = x^T A^4 x \) 的正惯性指数为 \textbf{\_\_}
\item
  已知随机变量 \( X \) 满足 \( E(X) = 2 \), \( E(X^2) = 8 \),
  \( E(X^4) = 160 \),则 \( X \) 的峰度系数为 \textbf{\_\_}
\item
  微分方程 \( y''' - 2y'' + y' = x e^x \sin x \) 的特解形式应设为
  \( x^k e^x (A\cos x + B\sin x) \),其中 \( k = \) \textbf{\_\_}
\item
  幂级数
  \( \sum_{n=1}^{\infty} \frac{(2n)!}{(n!)^2} \left(\int_0^1 x^n (1-x)^n dx\right) (x-2)^n \)
  的收敛区间为 \textbf{\_\_}
\item
  设 \( f(x,y,z) = e^{x+y+z} \),则 \( f \) 在点 \( (0,0,0) \) 处沿曲面
  \( \frac{x^2}{1} + \frac{y^2}{2} + \frac{z^2}{3} = 1 \)
  外法线方向的方向导数为 \textbf{\_\_}
\end{enumerate}

\subsection{三、解答题(共130分)}\label{ux4e09ux89e3ux7b54ux9898ux5171130ux5206uxff09}

15.(12分)设函数
\( f(x) = \int_0^x e^{-t^2} dt \cdot \int_0^x t^2 e^{-t^2} dt \),\\
(i) 证明 \( f(x) \) 满足微分方程
\( f''(x) + 4x^2 f(x) = 2x e^{-2x^2} \);\\
(ii) 求 \( f^{(2n)}(0) \) 的递推公式及 \( f^{(4)}(0) \) 的值。

16.(12分)已知矩阵 \( A = \begin{pmatrix} 
0 & 1 & 0 & 0 \\
0 & 0 & 1 & 0 \\
0 & 0 & 0 & 1 \\
1 & 0 & 0 & 0 
\end{pmatrix} \),\\
(i) 求 \( A \) 的特征值和最小多项式;\\
(ii) 计算 \( A^{100} \) 及 \( \det(e^A - I) \)。

17.(12分)设二维随机变量 \( (X,Y) \) 的联合概率密度为\\
\(
f(x,y) = \begin{cases} 
k(x^4 + y^4), & x^2 + y^2 \leq 1 \\
0, & 其他 
\end{cases}
\)\\
(i) 求常数 \( k \)(利用贝塔函数
\( B(m,n) = \int_0^1 t^{m-1}(1-t)^{n-1}dt \));\\
(ii) 求 \( X \) 与 \( Y \) 的相关系数 \( \rho_{XY} \);\\
(iii) 求 \( P(X^2 + Y^2 \leq \frac{1}{2}) \)。

18.(12分)求微分方程 \( y'' + 4y = \frac{1}{\sin 2x} \)
的通解,并讨论其在区间 \( (0,\frac{\pi}{2}) \) 内的所有有界解。

19.(14分)计算曲面积分 \( \iint_{\Sigma} (x^4 + y^4 + z^4) dS \),其中
\( \Sigma \) 为曲面 \( z = \sqrt{x^2 + y^2} \) 被平面 \( z = 1 \) 和
\( z = 2 \) 所截部分。

20.(14分)设级数 \( \sum_{n=1}^{\infty} a_n \) 绝对收敛,且
\( \sum_{n=1}^{\infty} n|a_n| \) 发散,证明:\\
(i) \( \sum_{n=1}^{\infty} \frac{a_n}{\ln(n+1)} \) 收敛;\\
(ii) \( \sum_{n=1}^{\infty} \frac{a_n}{n^p} \) 在 \( p > 0 \)
时绝对收敛,在 \( p \leq 0 \) 时可能发散。

21.(14分)设向量空间 \( V = \mathbb{R}^{3 \times 3} \),定义线性变换
\( T: V \to V \) 为 \( T(A) = \frac{1}{2}(A + A^T) \),\\
(i) 证明 \( T \) 是幂等变换(\( T^2 = T \));\\
(ii) 求 \( T \) 的特征值和特征子空间;\\
(iii) 若 \( A \in V \),证明 \( A \) 可唯一表示为 \( A = B + C \),其中
\( B \in \ker(T) \), \( C \in \text{im}(T) \)。

22.(14分)设总体 \( X \) 的概率密度为

\[f(x;\theta) = \begin{cases} 
\frac{1}{2\theta} e^{-\frac{|x|}{\theta}}, & x \in \mathbb{R} \\
0, & 其他 
\end{cases}\]

其中 \( \theta > 0 \) 为未知参数,\( X_1, X_2, \dots, X_n \) 为样本,\\
(i) 证明 \( \hat{\theta}_1 = \frac{1}{n} \sum_{i=1}^n |X_i| \) 是
\( \theta \) 的无偏估计量;\\
(ii) 求 \( \hat{\theta}_1 \) 的方差 \( D(\hat{\theta}_1) \);\\
(iii) 构造 \( \theta \) 的一个置信水平为 \( 1 - \alpha \) 的置信区间。

23.(16分)设函数 \( f(x,y) = \frac{1}{(1 - xy)^2} \),\\
(i) 求 \( f(x,y) \) 在点 \( (0,0) \) 处的n阶偏导数
\( \frac{\partial^n f(0,0)}{\partial x^k \partial y^{n-k}} \);\\
(ii) 证明 \( f(x,y) \) 在 \( |x| < 1, |y| < 1 \) 内满足拉普拉斯方程
\( \frac{\partial^2 f}{\partial x^2} + \frac{\partial^2 f}{\partial y^2} = 0 \);\\
(iii) 计算二重积分 \( \iint_{D} f(x,y) \ln(xy) dxdy \),其中
\( D = [0,\frac{1}{2}] \times [0,\frac{1}{2}] \)。

\end{document}
