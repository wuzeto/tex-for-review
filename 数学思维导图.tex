\documentclass[UTF8]{ctexart}
\usepackage{amsmath}
\usepackage{geometry}
\usepackage{amssymb}
\usepackage{graphicx}
\usepackage{booktabs}
\usepackage{extarrows}
\usepackage{makecell}
\usepackage{hyperref}
\usepackage{amsmath}
\usepackage{amssymb}
\usepackage{pgfplots}
\usepackage{amsthm}

\theoremstyle{remark}
\newtheorem{remark}{注}

\pgfplotsset{compat=1.18}  % 设置兼容版本(推荐)
\usepackage{tikz}
\usepackage{tikz}
\usetikzlibrary{shapes, arrows.meta, positioning, chains, decorations.pathreplacing, calligraphy}
\usepackage[dvipsnames, svgnames, x11names]{xcolor}
\usepackage{tcolorbox}
\usepackage[normalem]{ulem}
% 设置页面边距
\geometry{a4paper, margin=2cm}
\usepackage{amsmath}
\usepackage{amssymb}
\usepackage{tikz}
\usetikzlibrary{shapes, arrows.meta, positioning, chains, decorations.pathreplacing, calligraphy}
\usepackage[dvipsnames, svgnames, x11names]{xcolor}
\usepackage{tcolorbox}
\usepackage[normalem]{ulem}

\title{第5讲、第15讲 微分方程思维导图}
\author{}
\date{}

\begin{document}
	\maketitle
	
	\section{微分方程的概念}
	\subsection{微分方程及其阶}
	基本未知函数及其导数(或微分)与自变量之间关系的方程,方程中未知函数的最高阶导数的阶数称为微分方程的阶,“自变量与函数及其导数之间的关系”
	
	\subsection{常微分方程}
	未知函数是一元函数的微分方程
	
	\subsection{线性常微分方程}
	形如 \(a_n(x)y^{(n)} + a_{n - 1}(x)y^{(n - 1)} + \cdots + a_1(x)y' + a_0(x)y = f(x)\) 的微分方程称为线性常微分方程;当 \(f(x) \equiv 0\) 时,叫做齐次线性微分方程,否则叫做非齐次线性微分方程
	
	\subsection{微分方程的解}
	满足微分方程的函数,使方程成为恒等式,叫做微分方程的解;含有任意常数且任意常数的个数等于微分方程阶数的解,叫做微分方程的通解
	
	\subsection{微分方程的通解}
	若微分方程的解中含有相互独立的任意常数,且任意常数的个数等于微分方程的阶数,这样的解叫做微分方程的通解
	
	\subsection{初始条件与特解}
	确定通解中任意常数的条件称为初始条件,如 \(y(x_0) = y_0\),\(y'(x_0) = y_1\),\(\cdots\) ;确定了通解中的任意常数,就得到了特解
	
	\section{一阶微分方程的求解}
	\subsection{可分离变量微分方程}
	(1) 可分离变量:形如 \(y' = f(x)g(y)\) 的方程,通解为 \(\int \frac{1}{g(y)} dy = \int f(x) dx\) 
	
	(2) 齐次微分方程:形如 \(\frac{dy}{dx} = f\left(\frac{y}{x}\right)\) 的方程(可令 \(u = \frac{y}{x}\),则 \(y = xu\),\(\frac{dy}{dx} = u + x\frac{du}{dx}\) )
	
	\subsection{齐次型微分方程}
	一切如 \(\frac{dy}{dx} = \varphi\left(\frac{y}{x}\right)\) 的方程,求解时令 \(u = \frac{y}{x}\),则 \(y = xu\),\(\frac{dy}{dx} = u + x\frac{du}{dx}\) ,方程化为 \(u + x\frac{du}{dx} = \varphi(u)\) ,分离变量得 \(\int \frac{1}{\varphi(u) - u} du = \int \frac{1}{x} dx\) 
	
	\subsection{一阶线性微分方程}
	一切如 \(y' + P(x)y = Q(x)\) 的方程,通解公式为 \(y = e^{-\int P(x) dx} \left( \int Q(x) e^{\int P(x) dx} dx + C \right)\) 
	
	\subsubsection{伯努利方程}
	形如 \(\frac{dy}{dx} + P(x)y = Q(x)y^n\)(\(n \neq 0,1\)),通过令 \(z = y^{1 - n}\) 转化为一阶线性微分方程
	
	\subsection{二阶可降阶微分方程(仅列常见形式)}
	(1) \(y'' = f(x,y')\)(缺 \(y\) 型):令 \(p = y'\),则 \(y'' = p'\),方程化为 \(p' = f(x,p)\) 
	
	(2) \(y'' = f(y,y')\)(缺 \(x\) 型):令 \(p = y'\),则 \(y'' = p\frac{dp}{dy}\),方程化为 \(p\frac{dp}{dy} = f(y,p)\) 
	
	(3) \(y'' = f(x)\)(最简单的二阶方程,逐次积分):第一次积分得 \(y' = \int f(x) dx + C_1\),第二次积分得 \(y = \int \left( \int f(x) dx + C_1 \right) dx + C_2\) 
	
	\section{全微分方程}
	若 \(P(x,y)dx + Q(x,y)dy = 0\) 在区域 \(D\) 上是一阶全微分方程,且在 \(D\) 内某一单连通区域内 \(P,Q\) 具有连续的一阶偏导数,且 \(\frac{\partial P}{\partial y} = \frac{\partial Q}{\partial x}\) ,则方程为全微分方程,其通解为 \(u(x,y) = \int_{(x_0,y_0)}^{(x,y)} P(x,y)dx + Q(x,y)dy = C\) ,也可通过凑微分法求解
	
	\section{高阶线性微分方程的求解}
	(此处按思维导图中提及的框架,可结合具体内容补充,因原图未详细展开高阶线性方程具体解法步骤,先保留结构)
	
	\section{微分方程的几何应用}
	“强大的工具”——曲线(结合具体应用场景,因原图未详细展开,保留结构,可后续补充如根据曲线的切线、曲率等条件建立微分方程求解曲线方程等内容 )
	
	\section{微分方程的物理应用}
	(同理,按思维导图框架保留,可补充如根据运动学、动力学、热传导等物理规律建立微分方程求解物理量变化等内容 )
	
	% 以下为尝试用TikZ绘制思维导图结构示例(因手动精确还原全部手绘思维导图较复杂,先给出大致框架示意,可根据实际需求细化)
	\begin{tikzpicture}[
		node distance = 1.5cm,
		every node/.style = {draw, rectangle, rounded corners, align=center, minimum width=2cm, minimum height=1cm},
		arrow/.style = {->, >=Latex}
		]
		% 根节点
		\node (root) {微分方程};
		
		% 一级子节点
		\node (concept) [below of=root, xshift=-4cm] {微分方程的概念};
		\node (solve1) [below of=root, xshift=0cm] {一阶微分方程求解};
		\node (solve2) [below of=root, xshift=4cm] {高阶方程等};
		\node (appGeo) [below of=root, xshift=-2cm, yshift=-3cm] {几何应用};
		\node (appPhy) [below of=root, xshift=2cm, yshift=-3cm] {物理应用};
		
		% 连接根节点与一级子节点
		\draw[arrow] (root) -- (concept);
		\draw[arrow] (root) -- (solve1);
		\draw[arrow] (root) -- (solve2);
		\draw[arrow] (root) -- (appGeo);
		\draw[arrow] (root) -- (appPhy);
		
		% 二级子节点(以微分方程的概念下子项为例,继续细化)
		\node (order) [below of=concept, xshift=-3cm] {微分方程及其阶};
		\node (ode) [below of=concept, xshift=0cm] {常微分方程};
		\node (lode) [below of=concept, xshift=3cm] {线性常微分方程};
		\draw[arrow] (concept) -- (order);
		\draw[arrow] (concept) -- (ode);
		\draw[arrow] (concept) -- (lode);
		
		% 可继续类似方式细化更多子节点...
		
	\end{tikzpicture}
	
\end{document}