\documentclass[UTF8]{ctexart}
\usepackage{amsmath}
\usepackage{geometry}
\usepackage{amssymb}
\usepackage{graphicx}
\usepackage{booktabs}
\usepackage{extarrows}
\usepackage{multirow}
\usepackage{makecell}
\usepackage{hyperref}
\usepackage{amsmath}
\usepackage{amssymb}
\usepackage{pgfplots}
\usepackage{mathdots}
\usepackage{float}
\usepackage{amsthm}
\usepackage{bm}
\usepackage{esint}
\theoremstyle{remark}
\newtheorem{remark}{注}

\pgfplotsset{compat=1.18}  % 设置兼容版本(推荐)
\usepackage{tikz}
\usepackage{tikz}
\usetikzlibrary{shapes, arrows.meta, positioning, chains, decorations.pathreplacing, calligraphy}
\usepackage[dvipsnames, svgnames, x11names]{xcolor}
\usepackage{tcolorbox}
\usepackage[normalem]{ulem}
% 设置页面边距
\geometry{a4paper, margin=2cm}

\title{数一提升笔记}
\author{}
\date{}


\begin{document}
	\maketitle
	\tableofcontents
	\newpage
	
	\section{第一讲~~~函数}
	1. \textbf{函数}的定义:映射\(f:D\subset\mathbb{R}^n\rightarrow\mathbb{R}\)称为定义在\(D\)上的\(n\)元函数,记作\(u = f(\boldsymbol{x}),\boldsymbol{x}\in D\)。
	
	当\(n = 1\)时,\(y = f(x),x\in D\subset\mathbb{R}\)称为一元函数;
	
	当\(n = 2\)时,\(z = f(x,y),(x,y)\in D\subset\mathbb{R}^2\)称为二元函数。
	
	二元或二元以上的函数称为多元函数。
	
	2. 函数的两要素为定义域与对应法则;函数的表示方法主要有表格法、图形法和解析法。
	
	3. 一元函数\(y = f(x),x\in D\)的图形是平面点集\(C=\{(x,y)|y = f(x),x\in D\}\subset D\times f(D)\),通常是一条平面曲线。
	
	二元函数\(z = f(x,y),(x,y)\in D\)的图形通常是一张空间的曲面。
	
	4. 几个特殊类型的一元函数:
	\begin{enumerate}
		\item 绝对值函数:\(y = |x|=\begin{cases}x, & x\geq0\\-x, & x<0\end{cases}\)
	\item 符号函数:\(y = \mathrm{sgn}x=\begin{cases}1, & x>0\\0, & x = 0\\-1, & x<0\end{cases}\)
	\item 取整函数:\(y = [x]=n,n\leq x < n + 1,n = 0,\pm1,\pm2,\cdots\),表示不超过\(x\)的最大整数。
	\item Dirichlet 函数:\(y = D(x)=\begin{cases}1, & \text{当 }x\text{ 是有理数时}\\0, & \text{当 }x\text{ 是无理数时}\end{cases}\)
	\item 取最大值函数\(y = \max\{f(x),g(x)\}\),取最小值函数\(y = \min\{f(x),g(x)\}\)。
	\end{enumerate}
	
	5. 取整函数的性质:\([x]\leq x < [x]+1\)或\(x - 1 < [x]\leq x\)。
	
	6. 设有函数\(f(x)(x\in A)\)和\(g(x)(x\in B)\),且\(A\cap B\)非空,分别定义它们的加(减)法\(f\pm g\),乘法\(f\cdot g\)和除法\(\frac{f}{g}\)为
	
	\begin{align*}
	(f\pm g)(x)&=f(x)\pm g(x),\quad x\in A\cap B\\
	(f\cdot g)(x)&=f(x)\cdot g(x),\quad x\in A\cap B\\
	\left(\frac{f}{g}\right)(x)&=\frac{f(x)}{g(x)},\quad x\in A\cap B-\{x|g(x)=0\}
	\end{align*}
	
	7. 一元复合函数:设有函数链\(u = g(x),x\in D\subset\mathbb{R}\)及\(y = f(u),u\in D_1\subset\mathbb{R}\),若\(g(D)\subset D_1\),称\(y = f[g(x)](x\in D)\)为\(f\)与\(g\)确定的复合函数,记为\(f\circ g\),其中\(u\)称为中间变量。
	
	8. 初等函数:由基本初等函数与常值函数经过有限次四则运算或有限次复合运算得到的由一个统一的解析式表示的函数。
	
	9. 一元函数的反函数:若函数\(f:D\rightarrow f(D)\)是单射,则存在逆映射\(f^{-1}:f(D)\rightarrow D\),称此逆映射\(f^{-1}\)为\(f\)的反函数,且有\(f^{-1}(f(x)) = x\)和\(f(f^{-1}(x)) = x\)。
	
	10. 一元反函数\(f^{-1}\)的定义域就是函数\(f\)的值域,一元反函数\(f^{-1}\)的值域就是函数\(f\)的定义域,且函数\(y = f(x)\)和反函数\(y = f^{-1}(x)\)的图形关于直线\(y = x\)对称。
	
	11. 严格单调的一元函数必有反函数,且\(f\)严格单调递增(减)时,\(f^{-1}\)也严格单调递增(减)。
	
	12. 函数\(f(x)\)定义在对称区间\((-l,l)\)上,如果\(\forall x\in(-l,l)\),有\(f(-x)=-f(x)\),称\(f\)为\((-l,l)\)上的奇函数,若\(f(-x)=f(x)\),称\(f\)为\((-l,l)\)上的偶函数。
	
	13. 若\(f\)是奇函数,且存在反函数\(f^{-1}\),则\(f^{-1}\)也是奇函数。
	
	14. 函数\(f\)定义在\(\mathbb{R}\)上,若\(\exists T\in\mathbb{R}\),对\(\forall x\in\mathbb{R}\),有\(f(x + T)=f(x)\),称\(f\)为周期函数,\(T\)称为\(f\)的周期。
	
	15. 若\(T\)是\(f\)的周期,则\(T\)的任意整数倍\(mT\)也是\(f\)的周期。
	
	16. 若\(T_1,T_2\)是\(f\)的周期,则\(T_1\pm T_2\)也是\(f\)的周期。
	
	17. 周期函数的和与差不一定是周期函数。
	
	18. 任意周期函数不一定存在最小正周期,但非常数的连续周期函数存在最小正周期。
	
	19. 常用恒等式:
	\begin{enumerate}
		\item \(\arcsin x+\arccos x=\frac{\pi}{2},x\in[-1,1]\);
	\item \(\arctan x+\text{arccot }x=\frac{\pi}{2},x\in(-\infty,+\infty)\);
	\item \(\arctan x+\arctan\frac{1}{x}=\frac{\pi}{2},x\neq0\)。
	\end{enumerate}
	
	20. 求解函数方程的常用方法:
	\begin{enumerate}
	\item 利用初等变换求函数表达式;
	\item 利用函数的连续性求函数表达式;
	\item 利用导数的概念和性质求函数表达式;
	\item 利用不定积分和定积分的概念和性质求函数表达式;
	\end{enumerate}
	
	\section{第二讲~~~数列的极限}
	1. \textbf{确界}的定义:设\(E\)为实数集,\(\beta\in\mathbb{R}\),满足(1)\(\forall x\in E,x\leq\beta\),(2)\(\forall\varepsilon>0,\exists x_0\in E\),使得\(x_0>\beta - \varepsilon\),则称\(\beta\)为数集\(E\)的上确界,记作\(\beta = \sup E\),同理可以定义\(\alpha\)为数集\(E\)的下确界,记作\(\alpha = \inf E\)。
	
	2. 连续性公理:有上(下)界的数集一定存在上(下)确界。
	
	3. \textbf{数列极限}的定义:\(\lim_{n\rightarrow\infty}a_n = a\Leftrightarrow\forall\varepsilon>0,\exists N\in\mathbb{N}^+\),当\(n > N\)时,有\(|a_n - a|<\varepsilon\)。
	
	4. 若数列\(\{a_n\}\)存在极限,则该数列的极限唯一。
	
	5. 若数列\(\{a_n\}\)存在极限,则该数列一定有界,即\(\exists M>0\),使得\(|a_n|\leq M(n = 1,2,\cdots)\)。
	
	6. 若数列\(\{a_n\}\)存在极限\(a\),且\(a>0(a < 0)\),则\(\exists N\in\mathbb{N}^+\),当\(n > N\)时,有\(a_n>0(a_n < 0)\)。
	
	7. 若极限\(\lim_{n\rightarrow\infty}a_n = a\),且\(a_n\geq0(n = 1,2,\cdots)\),则\(a\geq0\)。
	
	8. 若极限\(\lim_{n\rightarrow\infty}a_n = a\),且\(a\neq0\),则\(\exists N\in\mathbb{N}^+\),当\(n > N\)时,恒有\(|a_n|>\frac{|a|}{2}\)。
	
	9. 若\(\lim_{n\rightarrow\infty}a_n = a\),则\(\lim_{n\rightarrow\infty}|a_n| = |a|\),反之不一定成立。
	
	10. \(\lim_{n\rightarrow\infty}|a_n| = 0\Leftrightarrow\lim_{n\rightarrow\infty}a_n = 0\)。
	
	11. 设\(a_n>0(n = 1,2,\cdots)\),且\(\lim_{n\rightarrow\infty}a_n = a>0\),则\(\lim_{n\rightarrow\infty}\sqrt{a_n}=\sqrt{a}\)。

	12. 若数列\(\{a_n\}\)有界,又\(\lim_{n\rightarrow\infty}b_n = 0\),则\(\lim_{n\rightarrow\infty}a_nb_n = 0\)。
	
	13. 若数列\(\{a_n\}\)和\(\{b_n\}\)存在极限,则\(\{a_n\pm b_n\}\),\(\{a_nb_n\}\)和\(\left\{\frac{a_n}{b_n}\right\}(\lim_{n\rightarrow\infty}b_n\neq0)\)均存在极限,且
	\begin{enumerate}
		\item \(\lim_{n\rightarrow\infty}(a_n\pm b_n)=\lim_{n\rightarrow\infty}a_n\pm\lim_{n\rightarrow\infty}b_n\);
		\item \(\lim_{n\rightarrow\infty}ka_n = k\lim_{n\rightarrow\infty}a_n\),其中\(k\)为常数;
		\item \(\lim_{n\rightarrow\infty}a_nb_n=\lim_{n\rightarrow\infty}a_n\cdot\lim_{n\rightarrow\infty}b_n\);
		\item \(\lim_{n\rightarrow\infty}\frac{a_n}{b_n}=\frac{\lim_{n\rightarrow\infty}a_n}{\lim_{n\rightarrow\infty}b_n}(\lim_{n\rightarrow\infty}b_n\neq0)\)。
	\end{enumerate}
	
	14. 数列\(\{a_n\}\)收敛,\(\{b_n\}\)发散,则\(\{a_n + b_n\}\)必发散;数列\(\{a_n\}\),\(\{b_n\}\)都发散,则数列\(\{a_n + b_n\}\)和\(\{a_nb_n\}\)可能收敛,也可能发散。
	15. 几个重要的数列极限:
	\begin{enumerate}
		\item \(\lim_{n\rightarrow\infty}\sqrt[n]{a}=1(a > 0)\)。
		\item \(\lim_{n\rightarrow\infty}\sqrt[n]{n}=1\)。
		\item 当\(|q| < 1\)时,\(\lim_{n\rightarrow\infty}q^n = 0\);当\(|q| > 1\)时,极限\(\lim_{n\rightarrow\infty}q^n\)不存在;\(\lim_{n\rightarrow\infty}(-1)^n\)不存在。
		\item \(\lim_{n\rightarrow\infty}(1 + \frac{1}{n})^n = \mathrm{e}\)。
		\item \(\lim_{n\rightarrow\infty}(1 + \frac{1}{2} + \frac{1}{3} + \cdots + \frac{1}{n} - \ln n)=\gamma\)(欧拉常数)。
	\end{enumerate}
	
	16. 夹逼准则:对\(\forall n\in\mathbb{N}^+\),若\(b_n\leq a_n\leq c_n\),且\(\lim_{n\rightarrow\infty}b_n = \lim_{n\rightarrow\infty}c_n = a\),则\(\lim_{n\rightarrow\infty}a_n = a\)。
	
	17. 单调有界准则:单调有界数列一定收敛。单调无界数列是确定符号的无穷大量。
	
	18. 闭区间套定理:对\(\forall n\in\mathbb{N}^+\),若闭区间序列\(\{[a_n,b_n]\}\)满足\(a_n\leq a_{n + 1}\leq b_{n + 1}\leq b_n\),则存在\(\xi\),使得\(a_n\leq\xi\leq b_n\)。如果\(\lim_{n\rightarrow\infty}|a_n - b_n| = 0\),则存在唯一\(\xi\),使得\(\lim_{n\rightarrow\infty}a_n = \lim_{n\rightarrow\infty}b_n = \xi\)。
	
	19. 凝聚定理(Weierstrass 定理):有界数列必有收敛子列。
	
	20. 若数列\(\{a_n\}\)收敛,则其任何子数列\(\{a_{n_k}\}\)也收敛,且\(\lim_{k\rightarrow\infty}a_{n_k} = \lim_{n\rightarrow\infty}a_n\)。
	
	21. 拉链定理:\(\lim_{n\rightarrow\infty}a_n = a\Leftrightarrow\lim_{n\rightarrow\infty}a_{2n} = \lim_{n\rightarrow\infty}a_{2n + 1} = a\)。
	
	22. \(p\)拉链定理:\(\lim_{n\rightarrow\infty}a_n = a\Leftrightarrow\lim_{n\rightarrow\infty}a_{pn} = \lim_{n\rightarrow\infty}a_{pn + 1} = \cdots = \lim_{n\rightarrow\infty}a_{pn + (n - 1)} = a(p\in\mathbb{N}^+)\)。
	
	23. 判定数列发散的方法:对于一个数列\(\{a_n\}\),如果能找到一个发散的子数列,则原数列发散;如果能找到两个子数列收敛于不同的极限,则原数列也发散。
	
	24. Cauchy 收敛准则:\(\lim_{n\rightarrow\infty}a_n = a\Leftrightarrow\forall\varepsilon>0,\exists N\in\mathbb{N}^+\),对\(\forall n,m > N\),有\(|a_n - a_m| < \varepsilon\)。或\(\lim_{n\rightarrow\infty}a_n = a\Leftrightarrow\forall\varepsilon>0,\exists N\in\mathbb{N}^+\),对\(\forall n > N\)和\(\forall p\in\mathbb{N}^+\),有\(|a_{n + p} - a_n| < \varepsilon\)。
	
	25. Cauchy 命题:若\(\lim_{n\rightarrow\infty}a_n = a\),则\(\lim_{n\rightarrow\infty}\frac{a_1 + a_2 + \cdots + a_n}{n}=a\)。
	
	26. 设\(a_n>0\),若\(\lim_{n\rightarrow\infty}\frac{a_{n + 1}}{a_n}=a\),则\(\lim_{n\rightarrow\infty}\sqrt[n]{a_n}=a\)。
	
	27. 若\(\lim_{n\rightarrow\infty}a_n = A\),\(\lim_{n\rightarrow\infty}b_n = B\),则\(\lim_{n\rightarrow\infty}\frac{a_1b_n + a_2b_{n - 1} + \cdots + a_nb_1}{n}=AB\)。
	
	28. \(\frac{*}{\infty}\)型的 Stolz 定理:设\(\{b_n\}\)为严格单调增加的数列,\(\lim_{n\rightarrow\infty}b_n = +\infty\),若\(\lim_{n\rightarrow\infty}\frac{a_n - a_{n - 1}}{b_n - b_{n - 1}} = A\),则\(\lim_{n\rightarrow\infty}\frac{a_n}{b_n}=A\)(\(A\)为有限数或\(\pm\infty\))。
	
	29. \(\frac{0}{0}\)型的 Stolz 定理:设\(\{b_n\}\)为严格单调减少的数列,\(\lim_{n\rightarrow\infty}a_n = \lim_{n\rightarrow\infty}b_n = 0\),若\(\lim_{n\rightarrow\infty}\frac{a_n - a_{n - 1}}{b_n - b_{n - 1}} = A\),则\(\lim_{n\rightarrow\infty}\frac{a_n}{b_n}=A\)(\(A\)为有限数或\(\pm\infty\))。
	
	30. 压缩映射:设函数\(f\)在\([a,b]\)上有定义,\(f([a,b])\subset[a,b]\),且存在常数\(r(0 < r < 1)\),使得对\(\forall x,y\in[a,b]\),都有\(|f(x) - f(y)|\leq r|x - y|\)成立,则称\(f\)为\([a,b]\)上的一个压缩映射,称常数\(r\)为压缩常数。
	
	31. 若\(f(x)\)为\([a,b]\)上的可微函数,且\(|f'(x)|\leq r < 1\),则\(f\)是\([a,b]\)上的一个压缩映射。
	
	32. 不动点:设函数\(f(x)\)在\([a,b]\)上有定义,若存在\(a^*\in[a,b]\),使得\(a^* = f(a^*)\),则称\(a^*\)为\(f(x)\)在\([a,b]\)上的一个不动点。
	
	33. 压缩映射原理(不动点原理):设\(f\)是\([a,b]\)上的一个压缩映射,则
	\begin{enumerate}
		\item \(f(x)\)在\([a,b]\)上存在唯一不动点\(a^*\);
		\item 对任意初始值\(a_1\in[a,b]\)和递推公式\(a_{n + 1} = f(a_n)(n = 1,2,\cdots)\)产生的数列\(\{a_n\}\)收敛于\(f(x)\)在\([a,b]\)上的唯一不动点\(a^*\);
		\item 事后估计\(|a_n - a^*|\leq\frac{r}{1 - r}|a_n - a_{n - 1}|\)和先验估计\(|a_n - a^*|\leq\frac{r^{n - 1}}{1 - r}|a_2 - a_1|\)成立。
	\end{enumerate}
	
	34. 设函数\(f(x)\)在区间\([a,b]\)上可积,则
	\begin{enumerate}
		\item 左和极限:\(\int_{a}^{b}f(x)dx=\lim_{n\rightarrow\infty}\frac{b - a}{n}\sum_{k = 1}^{n}f\left(a+\frac{k - 1}{n}(b - a)\right)\);
		\item 右和极限:\(\int_{a}^{b}f(x)dx=\lim_{n\rightarrow\infty}\frac{b - a}{n}\sum_{k = 1}^{n}f\left(a+\frac{k}{n}(b - a)\right)\);
		\item 中点和极限:\(\int_{a}^{b}f(x)dx=\lim_{n\rightarrow\infty}\frac{b - a}{n}\sum_{k = 1}^{n}f\left(a+\frac{2k - 1}{2n}(b - a)\right)\)。
	\end{enumerate}
	
	35. Wallis 公式:
	\begin{enumerate}
		\item \(\lim_{n\rightarrow\infty}\frac{1}{2n + 1}\left(\frac{2\cdot4\cdot\cdots\cdot(2n)}{1\cdot3\cdot\cdots\cdot(2n - 1)}\right)^2=\lim_{n\rightarrow\infty}\frac{1}{2n + 1}\left(\frac{(2n)!!}{(2n - 1)!!}\right)^2=\frac{\pi}{2}\);
		\item \(\frac{(2n)!!}{(2n - 1)!!}=\frac{2^{2n}(n!)^2}{(2n)!}\sim\sqrt{\pi n}(n\rightarrow\infty)\)。
	\end{enumerate}
	
	36. Stirling 公式:
	\begin{enumerate}
		\item \(n!=\sqrt{2\pi n}n^n\mathrm{e}^{-n+\frac{\theta_n}{12n}}\),其中\(0 < \theta_n < 1\);
		\item \(n!\sim\sqrt{2\pi n}\left(\frac{n}{\mathrm{e}}\right)^n(n\rightarrow\infty)\)。
	\end{enumerate}
	
	37. 数列极限的计算方法:
	\begin{enumerate}
		\item 利用数列极限的定义计算或证明数列极限;
		\item 利用初等变换和数列极限的四则运算法则计算数列极限;
		\item 利用夹逼准则计算数列极限;
		\item 利用单调有界准则计算数列极限;
		\item 利用函数极限计算方法和 Heine 定理来计算数列极限;
		\item 利用 Cauchy 收敛准则计算或证明数列极限;
		 \item 利用 Cauchy 命题计算或证明数列极限;
		\item 利用 Stolz 公式计算数列极限;
		\item 利用定积分的概念和性质计算或证明数列极限;
		\item 利用级数收敛的必要条件或性质计算数列极限。
	\end{enumerate}
	
	
\end{document}