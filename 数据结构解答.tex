\documentclass[UTF8]{ctexart}
\usepackage{amsmath, amssymb}
\usepackage{enumitem}
\usepackage{hyperref}
\usepackage{tasks}
\usepackage{graphicx}
\usepackage[a4paper, margin=0.6in]{geometry}
\usepackage{amsmath}
\usepackage{amssymb}
\usepackage{listings}
\usepackage{xcolor}
\usepackage{tikz}
\usepackage{amsmath}
\usepackage{amssymb}
\usepackage{ulem} 
\usepackage{fancyhdr}
\pagestyle{fancy}
\fancyhf{}
\rhead{数据结构黑皮书练习解答}
\lfoot{Page \thepage}
\cfoot{\today}

\definecolor{codegreen}{rgb}{0,0.6,0}
\definecolor{codegray}{rgb}{0.5,0.5,0.5}
\definecolor{codepurple}{rgb}{0.58,0,0.82}
\definecolor{backcolour}{rgb}{0.95,0.95,0.92}

\lstdefinestyle{mystyle}{
	backgroundcolor=\color{backcolour},   
	commentstyle=\color{codegreen},
	keywordstyle=\color{magenta},
	numberstyle=\tiny\color{codegray},
	stringstyle=\color{codepurple},
	basicstyle=\ttfamily\footnotesize,
	breakatwhitespace=false,         
	breaklines=true,                 
	captionpos=b,                    
	keepspaces=true,                 
	numbers=left,                    
	numbersep=5pt,                  
	showspaces=false,                
	showstringspaces=false,
	showtabs=false,                  
	tabsize=2
}

\lstset{style=mystyle}

\title{数据结构练习与解答}
\author{NgChakTung}
\date{\today}

\begin{document}
	\maketitle
	\tableofcontents
	\section{1.2~~函数与参数}
	\subsection{题目1:模板函数 Input}
	试编写一个模板函数Input,它要求用户输入一个非负数,并负责验证用户所输入的数是否真的大于或等于0,如果不是,它将告诉用户该输入非法,需要重新输入一个数。在函数非成功退出之前,应给用户三次机会。如果输入成功,函数应当把所输入的数作为引用参数返回。输入成功时,函数应返回true,否则返回false。上机测试该函数。
	
	要求用户输入一个非负数,并验证是否大于等于 0。失败时提示重新输入,最多三次机会。成功则返回 true 并通过引用返回值。
	
	\begin{lstlisting}[language=C++]
		#include <iostream>
		using namespace std;
		
		template<typename T>
		bool Input(T &value) {
			int attempts = 3;
			while (attempts--) {
				cout << "请输入一个非负数: ";
				cin >> value;
				if (!cin.fail() && value >= 0) {
					return true;
				}
				cin.clear();
				cin.ignore(10000, '\n');
				cout << "输入非法,请重新输入!剩余尝试次数:" << attempts << endl;
			}
			return false;
		}
	\end{lstlisting}
	
	\subsection{题目2:模板函数 isSorted}
	试编写一个模板函数,用来测试数组a中的元素是否按升序排列(即$a[i]\leq a[i+1]$,其中$0 \leq i \le n-1$)。如果不是,函数应返回false,否则应返回true。上机测试该函数。
	
	测试数组是否按升序排列。
	
	\begin{lstlisting}[language=C++]
		template<typename T>
		bool isSorted(const T a[], int n) {
			for (int i = 0; i < n - 1; ++i)
			if (a[i] > a[i + 1])
			return false;
			return true;
		}
	\end{lstlisting}
	
	\subsection{题目3:计算阶乘的非递归函数}
	试编写一个非递归函数来计算n!,并上机测试函数的正确性。
	
	计算 $ n! $
	
	\begin{lstlisting}[language=C++]
		unsigned long long factorial(int n) {
			if (n < 0) return 0;
			unsigned long long result = 1;
			for (int i = 2; i <= n; ++i)
			result *= i;
			return result;
		}
	\end{lstlisting}
	
	\subsection{题目4:斐波那契数列}
	1)试编写一个计算斐波那契数列Fn的递归函数,并上机测试其正确性。
	
	2)试说明对于任何n>2的整数,调用1)中的函数计算Fn时,同一个Fi会被处理至少一次。
	
	3)试编写一个非递归的函数来计算斐波那契数列Fn,该函数应能直接计算出每个斐波那契数。上机测试代码的正确性。
	
	\subsubsection{1) 递归版本}
	\begin{lstlisting}[language=C++]
		int fib_recursive(int n) {
			if (n == 0) return 0;
			if (n == 1) return 1;
			return fib_recursive(n - 1) + fib_recursive(n - 2);
		}
	\end{lstlisting}
	
	\subsubsection{2) 说明}
	对于任何 $ n > 2 $,递归调用会重复计算相同的子问题,例如:
	$$
	F_5 = F_4 + F_3 = (F_3 + F_2) + (F_2 + F_1)
	$$
	可以看到 $ F_2 $ 被多次计算。
	
	\subsubsection{3) 非递归版本}
	\begin{lstlisting}[language=C++]
		int fib_iterative(int n) {
			if (n == 0) return 0;
			if (n == 1) return 1;
			int a = 0, b = 1;
			for (int i = 2; i <= n; ++i) {
				int temp = a + b;
				a = b;
				b = temp;
			}
			return b;
		}
	\end{lstlisting}
	
	\subsection{题目5:输出所有子集}
	试编写一个递归函数,用来输出n个元素的所有子集。例如,三个元素\{a,b,c\}的所有子集是:\{\}(空集),$\{a\},\{b\},\{c\},\{a,b\},\{a,c\},\{b,c\}\text{和}\{a,b,c\}$。
	
	递归生成所有子集。
	
	\begin{lstlisting}[language=C++]
		#include <vector>
		#include <iostream>
		using namespace std;
		
		void printSubset(const vector<char>& subset) {
			cout << "{ ";
				for (char c : subset)
				cout << c << " ";
				cout << "}\n";
		}
		
		void generateSubsets(vector<char>& elements, int index, vector<char>& current) {
			if (index == elements.size()) {
				printSubset(current);
				return;
			}
			
			// 不选当前元素
			generateSubsets(elements, index + 1, current);
			
			// 选当前元素
			current.push_back(elements[index]);
			generateSubsets(elements, index + 1, current);
			current.pop_back(); // 回溯
		}
		
		void allSubsets(vector<char>& elements) {
			vector<char> current;
			generateSubsets(elements, 0, current);
		}
	\end{lstlisting}
	
	\subsection{题目6:判断元素 x 是否在数组中}
	试编写一个递归函数来确定元素x是否属于数组a[0:n-1]。
	
	递归实现查找。
	
	\begin{lstlisting}[language=C++]
		template<typename T>
		bool contains(T a[], int n, T x) {
			if (n <= 0) return false;
			if (a[0] == x) return true;
			return contains(a + 1, n - 1, x);
		}
	\end{lstlisting}
	
	\section{1.3~~~动态存储分配}
	
	\subsection{题目7:模板函数 ChangeSize1D}
	假定用一维数组a[0:size-1]来存储一组元素。如果有n个元素,可以把它们存储在a[0],...,a[n-1]中。当n超过size时,数组将不足以存储所有元素,必须分配一个更大的数组。类似地,如果元素的数目比size小很多,我们又可能希望减少数组的大小,以便释放出多余的空间为其他地方所用。试编写一个模板函数ChangeSize1D把数组a的大小从size变成ToSize。函数首先应该分配一个新的、大小为ToSize的数组,然后把原数组a中的n个元素复制到新数组a中,最后释放原数组a所占用的空间。上机测试该函数。
	
	编写一个模板函数,将一维数组从当前大小调整为新的大小。若新大小小于原大小,截断;若大于,则复制原有内容并在后面填充默认值。
	
	\begin{lstlisting}[language=C++]
		#include <iostream>
		#include <algorithm> // for std::min
		using namespace std;
		
		template<typename T>
		void ChangeSize1D(T*& a, int oldSize, int newSize) {
			if (newSize <= 0) {
				delete[] a;
				a = nullptr;
				return;
			}
			
			T* newArray = new T[newSize]; // 分配新数组
			int elementsToCopy = min(oldSize, newSize);
			for (int i = 0; i < elementsToCopy; ++i)
			newArray[i] = a[i];
			
			// 可选:如果 newSize > oldSize,可以初始化剩余空间为默认值
			for (int i = elementsToCopy; i < newSize; ++i)
			newArray[i] = T(); // 默认构造值(如 0、空等)
			
			delete[] a; // 释放旧数组
			a = newArray;
		}
	\end{lstlisting}
	
	\subsubsection{测试代码示例}
	\begin{lstlisting}[language=C++]
	int main() {
		int* arr = new int[5]{1, 2, 3, 4, 5};
		cout << "原始数组: ";
		for (int i = 0; i < 5; ++i) cout << arr[i] << " ";
		cout << endl;
		
		ChangeSize1D(arr, 5, 8);
		
		cout << "扩容后数组: ";
		for (int i = 0; i < 8; ++i) cout << arr[i] << " ";
		cout << endl;
		
		ChangeSize1D(arr, 8, 3);
		
		cout << "缩容后数组: ";
		for (int i = 0; i < 3; ++i) cout << arr[i] << " ";
		cout << endl;
		
		delete[] arr;
		return 0;
	}
\end{lstlisting}
	
	\subsection{题目8:函数 ChangeSize2D}试编写一个函数ChangeSize2D来改变一个二维数组的大小(见练习7)。上机测试该函数。
	
	
	改变二维数组的大小。与一维类似,但需要逐行处理。
	
	\begin{lstlisting}[language=C++]
		template<typename T>
		void ChangeSize2D(T& a, int oldRows, int oldCols, int newRows, int newCols) {
			T newArray = new T*[newRows]; // 分配新行指针
			
			for (int i = 0; i < newRows; ++i)
			newArray[i] = new T[newCols]; // 每行分配新列
			
			// 复制原数组数据
			for (int i = 0; i < min(oldRows, newRows); ++i)
			for (int j = 0; j < min(oldCols, newCols); ++j)
			newArray[i][j] = a[i][j];
			
			// 初始化新增位置为默认值(可选)
			for (int i = 0; i < newRows; ++i)
			for (int j = (i < oldRows ? oldCols : 0); j < newCols; ++j)
			newArray[i][j] = T();
			
			// 删除旧数组
			for (int i = 0; i < oldRows; ++i)
			delete[] a[i];
			delete[] a;
			
			a = newArray;
		}
	\end{lstlisting}
	
	\subsubsection{测试代码示例}
	\begin{lstlisting}[language=C++]
	int main() {
		int rows = 2, cols = 3;
		int matrix = new int*[rows];
		for (int i = 0; i < rows; ++i) {
			matrix[i] = new int[cols];
			for (int j = 0; j < cols; ++j)
			matrix[i][j] = i * cols + j + 1;
		}
		
		cout << "原始矩阵:" << endl;
		for (int i = 0; i < rows; ++i) {
			for (int j = 0; j < cols; ++j)
			cout << matrix[i][j] << " ";
			cout << endl;
		}
		
		ChangeSize2D(matrix, rows, cols, 3, 4);
		rows = 3, cols = 4;
		
		cout << "扩容后的矩阵:" << endl;
		for (int i = 0; i < rows; ++i) {
			for (int j = 0; j < cols; ++j)
			cout << matrix[i][j] << " ";
			cout << endl;
		}
		
		// 释放内存
		for (int i = 0; i < rows; ++i)
		delete[] matrix[i];
		delete[] matrix;
		
		return 0;
	}
\end{lstlisting}
	\section{1.4~~~类}
	\subsection{题目9:Currency 类的最大/最小表示范围分析}
	1)采用程序1-15中的描述,所能表示的最大和最小货币值分别是多少?假定用四个字节
	表示一个long型数据,用两个字节表示一个int型数据,则一个unsignedlong数介于0~232-1之间,一个unsignedint数介于0~65535之间。
	
	2)采用程序1-15中的描述,把dollars和cents变成int型,此时所能表示的最大和最小货币值分别是多少?
	
	3)如果用函数Add(见程序1-18)来累加两个货币值,为了确保从Currency类型转换成longint类型时不会发生错误,a1和a2最大可能的值应是多少?
	
	程序 1-15 中的 Currency 类描述如下:
	\begin{lstlisting}[language=C++]
		class Currency {
			private:
			long dollars;
			int cents;
		};
	\end{lstlisting}
	
	\subsubsection{1) long + int 表示范围}
	- 假设:
	- 'long' 占用 4 字节(32 位),有符号:$ -2^{31} \sim 2^{31}-1 $
	- 'int' 占用 2 字节(16 位),有符号:$ -32768 \sim 32767 $
	
	所以一个货币值最大为:
	$$
	\text{max} = 2^{31}-1 + \frac{9999}{100} = 21474836.99
	$$
	最小为:
	$$
	\text{min} = -2^{31} + \frac{-9999}{100} = -21474836.99
	$$
	
	\subsubsection{2) 使用 int 表示 dollars 和 cents}
	
	若改为:
	\begin{lstlisting}[language=C++]
		int dollars;
		int cents;
	\end{lstlisting}
	
	每个字段范围是 $-32768 \sim 32767$,因此最大值为:
	$$
	\text{max} = 32767.99
	$$
	最小值为:
	$$
	\text{min} = -32768.99
	$$
	
	\subsubsection{3) Add 函数安全性限制}
	
	函数原型:
	\begin{lstlisting}[language=C++]
		void Add(Currency a1, Currency a2);
	\end{lstlisting}
	
	内部使用了类型转换为 'long int'。为防止溢出,必须保证:
	
	\begin{align}
	a1.dollars + a1.cents / 100 < 2^{31}-1 \\
	a2.dollars + a2.cents / 100 < 2^{31}-1
		\end{align}
	
	即:
	
	\begin{align}
	a1.dollars \leq 21474836,\quad a1.cents \leq 99 \\
	a2.dollars \leq 21474836,\quad a2.cents \leq 99
\end{align}
	
	
	
	\subsection{题目10:扩充 Currency 类功能}
	试扩充程序1-15中的Currency类,为该类添加如下的public成员函数:
	
	1)Input()——从标准输入流中接收一个货币值,并把它返回给调用者。
	
	2)Subtract(x)——从当前对象中减去对象x的值,并把结果返回。
	
	3)Percent(x)——返回一个Currency对象,其值为当前对象的x\%,其中,x是一个浮点数。
	
	4)Multiply(x)——返回一个Currency对象,其值为当前对象乘以浮点数x。
	
	5)Devide(x)——返回一个Currency对象,其值为当前对象除以浮点数x。
	
	基于程序 1-15 扩展以下成员函数:
	
	\begin{lstlisting}[language=C++]
		class Currency {
			public:
			void Input();               // 输入货币值
			void Subtract(const Currency& x); // 减法
			Currency Percent(float x);  // 百分比
			Currency Multiply(float x); // 乘法
			Currency Divide(float x);   // 除法
			private:
			long dollars;
			int cents;
		};
	\end{lstlisting}
	
	\subsubsection{实现细节}
	\begin{lstlisting}[language=C++]
		void Currency::Input() {
			float value;
			std::cin >> value;
			dollars = static_cast<long>(value);
			cents = static_cast<int>((value - dollars) * 100 + 0.5); // 四舍五入
		}
		
		void Currency::Subtract(const Currency& x) {
			long totalCents = (dollars * 100 + cents) - (x.dollars * 100 + x.cents);
			dollars = totalCents / 100;
			cents = abs(totalCents % 100);
		}
		
		Currency Currency::Percent(float x) {
			long totalCents = (dollars * 100 + cents) * x / 100;
			Currency result;
			result.dollars = totalCents / 100;
			result.cents = totalCents % 100;
			return result;
		}
		
		Currency Currency::Multiply(float x) {
			long totalCents = (dollars * 100 + cents) * x;
			Currency result;
			result.dollars = totalCents / 100;
			result.cents = totalCents % 100;
			return result;
		}
		
		Currency Currency::Divide(float x) {
			long totalCents = (dollars * 100 + cents) / x;
			Currency result;
			result.dollars = totalCents / 100;
			result.cents = totalCents % 100;
			return result;
		}
	\end{lstlisting}

	
	\subsection{题目11:使用程序 1-21 的描述实现相同功能}
	采用程序1-21中的描述来完成练习10。
	
	如果程序 1-21 是将 'dollars' 和 'cents' 合并为一个 'long totalCents;' 来存储总美分数,则类定义如下:
	
	\begin{lstlisting}[language=C++]
		class Currency {
			public:
			void Input();
			void Subtract(const Currency& x);
			Currency Percent(float x);
			Currency Multiply(float x);
			Currency Divide(float x);
			private:
			long totalCents;
		};
	\end{lstlisting}
	
	\subsubsection{实现}
	\begin{lstlisting}[language=C++]
		void Currency::Input() {
			float value;
			std::cin >> value;
			totalCents = static_cast<long>(value * 100 + 0.5);
		}
		
		void Currency::Subtract(const Currency& x) {
			totalCents -= x.totalCents;
		}
		
		Currency Currency::Percent(float x) {
			Currency result;
			result.totalCents = totalCents * x / 100;
			return result;
		}
		
		Currency Currency::Multiply(float x) {
			Currency result;
			result.totalCents = totalCents * x;
			return result;
		}
		
		Currency Currency::Divide(float x) {
			Currency result;
			result.totalCents = totalCents / x;
			return result;
		}
	\end{lstlisting}
	
	\subsection{题目12:重载操作符和赋值运算符}
	1)采用程序1-24中的描述来完成练习10。重载操作符>>,-,\%,*和/。在重载操作符>>时,
	可把它定义成一个友元函数,不必专门定义一个public输入函数。
	
	2)利用重载赋值操作符=来替换两个Set函数。把一个整数赋值给一个Currency对象可用
	operator=(intx)来表示,它可用来替换第一个Set函数,其中x表示一个包含符号、美元和美分
	的整数。同样,operator=(floatx)可用来替换第二个Set函数。
	
	\subsubsection{1) 重载操作符}
	
	\begin{lstlisting}[language=C++]
		class Currency {
			public:
			friend std::istream& operator>>(std::istream&, Currency&);
			Currency operator-(const Currency&) const;
			Currency operator%(float) const;
			Currency operator*(float) const;
			Currency operator/(float) const;
			private:
			long totalCents;
		};
	\end{lstlisting}
	
	\begin{lstlisting}[language=C++]
		std::istream& operator>>(std::istream& in, Currency& c) {
			float val;
			in >> val;
			c.totalCents = static_cast<long>(val * 100 + 0.5);
			return in;
		}
		
		Currency Currency::operator-(const Currency& x) const {
			Currency result;
			result.totalCents = totalCents - x.totalCents;
			return result;
		}
		
		Currency Currency::operator%(float x) const {
			Currency result;
			result.totalCents = totalCents * x / 100;
			return result;
		}
		
		Currency Currency::operator*(float x) const {
			Currency result;
			result.totalCents = totalCents * x;
			return result;
		}
		
		Currency Currency::operator/(float x) const {
			Currency result;
			result.totalCents = totalCents / x;
			return result;
		}
	\end{lstlisting}
	
	\subsubsection{2) 重载赋值操作符}
	
	\begin{lstlisting}[language=C++]
		class Currency {
			public:
			Currency& operator=(int x);      // 整数赋值
			Currency& operator=(float x);    // 浮点数赋值
		};
		
		Currency& Currency::operator=(int x) {
			totalCents = x * 100;
			return *this;
		}
		
		Currency& Currency::operator=(float x) {
			totalCents = static_cast<long>(x * 100 + 0.5);
			return *this;
		}
	\end{lstlisting}
	\section{3.3~~~公式化描述}
	\subsection{练习1:动态扩容 LinearList 类}
	类LinearList的一个缺点是需要预测线性表最大可能的尺寸,解决的方法之一是:在创建线性表时,置MaxSize=1,之后在执行插入操作期间,如果在表中已经有了MaxSize个元素,则将MaxSize加倍,按照这个新尺寸分配一个新数组,并将老数组中的数据复制到新数组中,最后将老数组删除。类似地,在执行删除操作期间,如果线性表的尺寸降至当前MaxSize的四分之一,则分配一个更小的、尺寸为MaxSize/2的数组,并将老数组中的数据复制到新数组中,最后将老数组删除。
	
	1)采用上述思想,重新实现类LinearList。构造函数不应带参数,并将MaxSize置为1,并分配一个大小为1的数组,同时置length=0。
	
	2)考虑从一个空表开始,对其连续实施n次表操作。假定在使用以前的实现方法时用了f(n)步,试证明对于上述新的实现方法,存在常数c,使得执行步数最多为cf(n)。
	
	\subsubsection{1) 动态扩容类定义}
	
	\begin{lstlisting}[language=C++]
		template <typename T>
		class LinearList {
			public:
			LinearList() : length(0), MaxSize(1) { element = new T[MaxSize]; }
			~LinearList() { delete[] element; }
			void Insert(int i, const T& x);
			void Delete(int i);
			// 其他函数省略...
			private:
			int length;
			int MaxSize;
			T* element;
			void ChangeSize();
		};
	\end{lstlisting}
	\subsubsection{扩容/缩容逻辑}
	
	\begin{lstlisting}[language=C++]
		template <typename T>
		void LinearList<T>::ChangeSize() {
			int newSize = MaxSize;
			if (length == MaxSize) newSize *= 2;
			else if (MaxSize > 1 && length <= MaxSize / 4) newSize /= 2;
			else return;
			T* newArray = new T[newSize];
			for (int i = 0; i < length; ++i)
			newArray[i] = element[i];
			delete[] element;
			element = newArray;
			MaxSize = newSize;
		}
	\end{lstlisting}
	\subsubsection{插入/删除调用 ChangeSize}
	
	\begin{lstlisting}[language=C++]
		template <typename T>
		void LinearList<T>::Insert(int i, const T& x) {
			// 检查索引有效性...
			ChangeSize();
			// 插入元素...
		}
		template <typename T>
		void LinearList<T>::Delete(int i) {
			// 删除元素...
			ChangeSize();
		}
	\end{lstlisting}
	\subsubsection{2) 时间复杂度分析}
	
	假设原实现中每次插入操作需要 O(1) 时间,总操作数为 $ f(n) $。由于新实现中扩容操作总时间是 $ O(n) $,因此存在常数 $ c $,使得新方法的时间复杂度为 $ cf(n) $。
	\subsection{练习2:反序操作 Reverse}
	
	假设一个线性表的描述满足公式
	location(i)=i-1
	
	1)扩充LinearList类的定义,增加一个函数Reverse,该函数将表中元素的次序变反。反序操作是就地进行的(即在数组element本身的空间内)。注意,在反序操作进行之前,表中第k个元素(如果存在)位于element[k-1],完成反序之后,该元素位于element[length-k]。
	
	2)证明上述函数的复杂性与线性表的长度成线性关系。
	
	3)采用适当的测试数据来验证代码的正确性。
	
	4)请编写另外一个就地处理的反序函数,它能对LinearList类型的对象进行反序操作。该函数不是LinearList类的成员函数,但它应利用成员函数来产生反序线性表。
	
	5)上述函数的时间复杂性是多少?
	
	6)分别采用大小为1000,5000和10000的线性表来比较以上两种反序函数的执行效率。
	
	\subsubsection{1) 成员函数 Reverse()}
	
	\begin{lstlisting}[language=C++]
		template <typename T>
		void LinearList<T>::Reverse() {
			for (int i = 0; i < length / 2; ++i)
			swap(element[i], element[length - 1 - i]);
		}
	\end{lstlisting}
	\subsubsection{2) 复杂度分析}
	
	循环执行 $ \lfloor \frac{n}{2} \rfloor $ 次,每步为 O(1),总复杂度为 $ O(n) $。
	\subsubsection{3) 非成员函数 ReverseList()}
	
	\begin{lstlisting}[language=C++]
		template <typename T>
		void ReverseList(LinearList<T>& list) {
			for (int i = 0; i < list.length / 2; ++i)
			swap(list.element[i], list.element[list.length - 1 - i]);
		}
	\end{lstlisting}
	\subsubsection{5) 时间复杂度}
	
	两者均为 $ O(n) $。
	\subsubsection{6) 测试建议}
	
	分别对长度为 1000、5000、10000 的线性表进行多次运行测试,记录耗时并比较两种方式效率差异。
	\subsection{练习3:Half() 函数}
	
	扩充LinearList类的定义,增加一个成员函数Half()。调用X.Half()将删除X中半数的元素。比如,如果X.length的初始值为7,且X.element[]=[2,13,4,5,17,8,29],则执行X.Half()后,X.length的值为4,且X.element[]=[2,4,17,29];如果X.length的初始值为4,且X.element[]=[2,13,4,5],则执行X.Half()后,X.length的值为2,且X.element[]=[2,4]。如果X开始时是空表,则执行X.Half()后,X仍然为空表。
	
	1)试给出成员函数Half()的代码。不能利用任何其他的LinearList成员函数。代码的复杂性应为$\Theta$(length)。
	
	2)证明代码的复杂性确实为$\Theta$(length)。
	
	3)使用适当的测试数据来测试代码的正确性。
	
	\subsubsection{1) 实现 Half()}
	
	\begin{lstlisting}[language=C++]
		template <typename T>
		void LinearList<T>::Half() {
			int newLength = length / 2 + length % 2;
			int j = 0;
			for (int i = 0; i < length; i += 2)
			element[j++] = element[i];
			length = newLength;
		}
	\end{lstlisting}
	\subsubsection{2) 复杂度分析}
	
	遍历整个数组一次,每步执行 O(1),总复杂度为 $ \Theta(n) $。
	\subsubsection{3) 测试建议}
	
	使用示例:
	
	- 原始数组: [2,13,4,5,17,8,29] → [2,4,17,29]
	
	- 原始数组: [2,13,4,5] → [2,4]
	\subsection{练习4:复制构造函数}
	LinearList缺省的复制构造函数仅复制length,MaxSize和element的值。因此,当用线性表L作为函数F的实际参数(对应于值参X)来调用F时,L.length,L.MaxSize和L.element被复制到
	X的相应成员中。当函数F退出时,LinearList的析构函数被X唤醒,数组X.element(与数组
	L.element相同)被删除。避免L.element被删除的一种方法是定义复制构造函数如下:
	LinearList<T>::LinearList(constLinearList<T>\&L)
	该函数复制length和MaxSize的值,然后创建一个新数组element,并将L.element[0:MaxSize-1]复制到element中。试编写这个复制构造函数,并估计其复杂性。
	
	\subsubsection{实现复制构造函数}
	
	\begin{lstlisting}[language=C++]
		template <typename T>
		LinearList<T>::LinearList(const LinearList<T>& L)
		: length(L.length), MaxSize(L.MaxSize) {
			element = new T[MaxSize];
			for (int i = 0; i < length; ++i)
			element[i] = L.element[i];
		}
	\end{lstlisting}
	\subsubsection{复杂度分析}
	
	复制所有元素,时间复杂度为 $ O(n) $。
	\subsection{练习5:游标操作}
	在许多应用中,需要对一个线性表中的元素进行前移和后移操作。试扩充LinearList类
	的定义,增加一个私有成员变量current,它纪录线性表当前的位置。此外,需要增加的共享成
	员如下:
	
	1)Reset—置current为1。
	
	2)Current(x)—返回x中的当前元素。
	
	3)End—当且仅当当前元素为表的最后一个元素时,返回true。
	
	4)Front—当且仅当当前元素为表的第一个元素时,返回true。
	
	5)Next—移动current至表中的下一个元素,如果操作失败则引发一个异常。
	
	6)Previous—移动current至表中的前一个元素,如果操作失败则引发一个异常。
	
	试编写上述代码,并使用适当的测试数据来测试代码的正确性。
	\subsubsection{类定义新增}
	
	\begin{lstlisting}[language=C++]
		private:
		int current; // 当前位置(从1开始)
			template <typename T>
			void LinearList<T>::Reset() { current = 1; }
			template <typename T>
			bool LinearList<T>::End() const { return current == length; }
			template <typename T>
			bool LinearList<T>::Front() const { return current == 1; }
			template <typename T>
			void LinearList<T>::Next() {
				if (current >= length) throw std::out_of_range("超出末尾");
				++current;
			}
			template <typename T>
			void LinearList<T>::Previous() {
				if (current <= 1) throw std::out_of_range("超出开头");
				--current;
			}
		\end{lstlisting}
		\subsection{练习6:交替合并两个线性表 Alternate(A,B)}
		设A和B均为LinearList对象
		
		1)编写一个新的成员函数Alternate(A,B)以创建一个新的线性表,该表包含了A和B中的所
		有元素,其中A和B的元素轮流出现,表中的首元素为A中的第一个元素。在轮流排列元素时,
		如果某个表的元素用完了,则把另一个表的其余元素依次添加在新表的后部。代码的复杂性应
		与两个输入表的长度呈线性比例关系。
		
		2)证明代码具有线性复杂性。
		
		3)使用适当的测试数据来测试代码的正确性。
		\subsubsection{实现 Alternate()}
		
		\begin{lstlisting}[language=C++]
			template <typename T>
			void LinearList<T>::Alternate(const LinearList<T>& A, const LinearList<T>& B) {
				Clear(); // 清空当前列表
				int lenA = A.length, lenB = B.length;
				int n = max(lenA, lenB);
				for (int i = 0; i < n; ++i) {
					if (i < lenA) Insert(length + 1, A.element[i]);
					if (i < lenB) Insert(length + 1, B.element[i]);
				}
			}
		\end{lstlisting}
		\subsubsection{复杂度分析}
		
		最多插入 $ 2n $ 次,每次插入平均 O(1),总复杂度为 $ O(n) $。
		\subsection{练习7:有序合并 Merge(A,B)}
		设A和B均为LinearList对象。假定A和B中的元素都是按序排列的(如从左至右按递增次
		序排列)。
		
		1)试编写一个成员函数Merge(A,B),用以创建一个新的有序线性表,该表中包含了A和B
		的所有元素。
		
		2)考察所编写的函数的时间复杂性。
		
		3)用适当的测试数据来测试代码的正确性。
		
		\subsubsection{实现 Merge()}
		
		\begin{lstlisting}[language=C++]
			template <typename T>
			void LinearList<T>::Merge(const LinearList<T>& A, const LinearList<T>& B) {
				Clear();
				int i = 0, j = 0;
				while (i < A.length && j < B.length) {
					if (A.element[i] <= B.element[j])
					Insert(length + 1, A.element[i++]);
					else
					Insert(length + 1, B.element[j++]);
				}
				while (i < A.length) Insert(length + 1, A.element[i++]);
				while (j < B.length) Insert(length + 1, B.element[j++]);
			}
		\end{lstlisting}
		\subsubsection{复杂度分析}
		
		每个元素仅处理一次,总复杂度为 $ O(m+n) $。
		\subsection{练习8:Split(A,B)}
		1)试编写函数LinearList::Split(A,B),该函数用来创建两个线性表A和B,A中包含*this
		的所有奇数元素(注意一个线性表的奇数元素是指element偶数位置上的元素),B中包含其余
		的元素。
		
		2)考察函数的时间复杂性。
		
		3)用适当的测试数据来测试代码的正确性。
		\subsubsection{实现 Split()}
		
		\begin{lstlisting}[language=C++]
			template <typename T>
			void LinearList<T>::Split(LinearList<T>& A, LinearList<T>& B) {
				A.Clear(); B.Clear();
				for (int i = 0; i < length; ++i) {
					if (i % 2 == 0) A.Insert(A.length + 1, element[i]);
					else B.Insert(B.length + 1, element[i]);
				}
			}
		\end{lstlisting}
		\subsubsection{复杂度分析}
		
		遍历一次数组,插入 O(1),总复杂度为 $ O(n) $。
		\subsection{练习9:location(i)=i 的实现}
		假定采用如下公式来描述一个线性表:
		
		$location(i)=i$(3-2)
		
		1)对于程序3-1中的类定义需要做相应的修改吗?如果是,请编写新的定义。
		
		2)能够描述的最长的表的长度是多少?
		
		3)修改程序3-1中的所有函数,以满足公式(3-2)
		
		4)用适当的测试数据来测试代码的正确性。
		
		5)每个函数的时间复杂性分别是多少?
		
		\subsubsection{1) 修改类定义}
		
		修改插入/删除逻辑,使 'element[i]' 对应第 'i+1' 个元素。
		\subsubsection{2) 最大长度}
		
		仍为 'MaxSize'。
		\subsubsection{3) 所有函数需相应调整索引处理。}
		
		以下是部分核心函数的修改示例:
		
		\begin{lstlisting}[language=C++]
			template <typename T>
			T LinearList<T>::Get(int i) const {
				if (i < 0 || i >= length)
				throw std::out_of_range("Index out of range");
				return element[i];
			}
			
			template <typename T>
			void LinearList<T>::Insert(int i, const T& x) {
				if (i < 0 || i > length)
				throw std::out_of_range("Index out of range");
				ChangeSize();
				for (int j = length; j > i; --j)
				element[j] = element[j - 1];
				element[i] = x;
				++length;
			}
			
			template <typename T>
			void LinearList<T>::Delete(int i) {
				if (i < 0 || i >= length)
				throw std::out_of_range("Index out of range");
				for (int j = i; j < length - 1; ++j)
				element[j] = element[j + 1];
				--length;
				ChangeSize();
			}
		\end{lstlisting}
			
			\subsection{练习14:循环队列式 LinearList}
			假定采用下述公式来描述一个线性表:
			$location(i)=(location(1)+i-1)\%MaxSize$(3-3)
			
			其中MaxSize是用来存储表元素的数组的大小。与专门保留一个表长的做法不同的是,用变量
			first和last来指出表的第一个元素和最后一个元素的位置。
			
			1)基于该公式,设计一个与LinearList相似的类。
			
			2)first和last的初始值应该是多少?
			
			3)对于新的类,试编写出所有成员函数的代码。(通过适当的选择,将元素移动到欲插入/
			删除元素的左边或右边,可以编写出更高效的Delete和Insert代码。)
			
			4)考察每个函数的时间复杂性。
			
			5)用适当的测试数据来测试代码的正确性。
			
			\subsubsection{1) 新类设计}
			使用 'first' 和 'last' 表示首尾位置,采用模运算定位。
			\begin{lstlisting}[language=C++]
				template <typename T>
				class CircularList {
					public:
					CircularList(int size = 10);
					~CircularList();
					
					void Insert(int pos, const T& x);
					void Delete(int pos);
					T Get(int pos) const;
					bool IsEmpty() const { return length == 0; }
					int Size() const { return length; }
					
					private:
					int MaxSize;
					int first;
					int last;
					int length;
					T* element;
				};
			\end{lstlisting}
			\subsubsection{2) 初始值设置}
			\begin{lstlisting}[language=C++]
			first = last = -1;
			length = 0;
		\end{lstlisting}
			\subsubsection{3) Insert/Delete 示例}
			\begin{lstlisting}[language=C++]
				template <typename T>
				void CircularList<T>::Insert(int pos, const T& x) {
					if (length == MaxSize)
					throw std::overflow_error("List is full");
					
					int realPos = (first + pos) % MaxSize;
					
					// 移动元素
					for (int i = last; i >= realPos; --i)
					element[(i + 1) % MaxSize] = element[i];
					
					element[realPos] = x;
					++length;
					last = (last + 1) % MaxSize;
					if (first == -1) first = 0;
				}
			\end{lstlisting}
			
			\begin{lstlisting}[language=C++]
				template <typename T>
				void CircularList<T>::Delete(int pos) {
					if (length == 0)
					throw std::underflow_error("List is empty");
					
					int realPos = (first + pos) % MaxSize;
					
					for (int i = realPos; i < last; ++i)
					element[i] = element[(i + 1) % MaxSize];
					
					--length;
					last = (last - 1 + MaxSize) % MaxSize;
					if (length == 0) first = last = -1;
				}
			\end{lstlisting}
			\subsubsection{4) 时间复杂度}
			插入/删除最坏情况 O(n),其余操作 O(1)
			\subsection{练习20:在多个线性表之间插入元素}
			编写一个C++函数,该函数在表i的第k个元素之后插入一个元素。假定在一个数组中
			存放了n个线性表。如果不得不移动多个表以便容纳新元素,新的函数应该首先确定可用空间
			的数量。移动表时应保证每个表所包含的可用空间数量应大体相同,以便适应未来增长的需要。
			通过编译和执行,测试代码的正确性。
			
			\subsubsection{思路}
			维护一个全局数组 'T* storage',以及每个线性表的起始位置和长度。插入时计算可用空间,必要时重新分配并平衡各表空间。
			\begin{lstlisting}[language=C++]
				template <typename T>
				void MultiList<T>::InsertAfter(int tableIndex, int k, const T& x) {
					if (tableIndex < 0 || tableIndex >= numTables)
					throw std::out_of_range("Invalid table index");
					
					Table& t = tables[tableIndex];
					if (t.length >= t.capacity)
					ExpandTable(tableIndex);
					
					int insertPos = t.start + k + 1;
					
					// 向后移动后续表的数据
					for (int i = numTables - 1; i > tableIndex; --i) {
						MoveData(tables[i].start + 1, tables[i].start, tables[i].length);
						tables[i].start += 1;
					}
					
					storage[insertPos] = x;
					t.length += 1;
				}
			\end{lstlisting}
			\subsection{练习21:从指定表中删除元素}
			编写一个C++函数,该函数用来从表i中删除第k个元素。假定在一个数组中存放了n
			个线性表。通过编译和执行,测试代码的正确性。
			\subsubsection{实现思路}
			根据表号和元素位置找到对应存储区域,移动元素,更新长度信息。
			\begin{lstlisting}[language=C++]
				template <typename T>
				void MultiList<T>::DeleteFrom(int tableIndex, int k) {
					if (tableIndex < 0 || tableIndex >= numTables)
					throw std::out_of_range("Invalid table index");
					
					Table& t = tables[tableIndex];
					if (k < 0 || k >= t.length)
					throw std::out_of_range("Invalid element index");
					
					int deletePos = t.start + k;
					
					// 删除元素
					for (int i = deletePos; i < t.start + t.length - 1; ++i)
					storage[i] = storage[i + 1];
					
					t.length -= 1;
					
					// 向前移动后续表数据
					for (int i = tableIndex + 1; i < numTables; ++i) {
						MoveData(tables[i].start - 1, tables[i].start, tables[i].length);
						tables[i].start -= 1;
					}
				}
			\end{lstlisting}
			
	\section{3.4~~~链表描述}
			
	\subsection*{练习22:复制构造函数 Chain<T>::Chain(const Chain<T>\&C)}
	编写一个复制构造函数Chain<T>::Chain(constChain<T>\&C),把链表C中的元素复制到新的节点中。这个构造函数的复杂性是多少?
	\subsubsection*{实现代码}
	
	\begin{lstlisting}[language=C++]
		template <typename T>
		class ChainNode {
			public:
			T data;
			ChainNode<T>* link;
		};
		
		template <typename T>
		class Chain {
			public:
			Chain() : first(nullptr), last(nullptr) {}
			Chain(const Chain<T>& C); // 复制构造函数
			~Chain();
			
			private:
			ChainNode<T>* first;
			ChainNode<T>* last;
		};
		
		template <typename T>
		Chain<T>::Chain(const Chain<T>& C) {
			if (C.first == nullptr) {
				first = last = nullptr;
				return;
			}
			
			ChainNode<T>* src = C.first;
			ChainNode<T>* dst = new ChainNode<T>;
			dst->data = src->data;
			first = dst;
			src = src->link;
			
			while (src != nullptr) {
				dst->link = new ChainNode<T>;
				dst = dst->link;
				dst->data = src->data;
				src = src->link;
			}
			dst->link = nullptr;
			last = dst;
		}
	\end{lstlisting}
	
	\subsubsection*{时间复杂度分析}
	
	遍历整个链表一次,每个节点复制一次。
	
	 时间复杂度:O(n)
	
	
	
	\subsection*{练习23:数组线性表转单向链表}
	编写一个函数,把一个用数组表示的线性表转换成单向链表。要求利用LinearList的成员
	函数Find和Chain的成员函数Insert来实现。该函数的时间复杂性是多少?试测试代码的正确性。
	
	使用 'LinearList' 的 'Find()' 和 'Chain' 的 'Insert()' 实现。
	
	\begin{lstlisting}[language=C++]
		template <typename T>
		void ArrayToChain(LinearList<T>& list, Chain<T>& chain) {
			for (int i = 1; i <= list.Size(); ++i) {
				T x;
				if (list.Find(i, x)) {
					chain.Insert(chain.Length() + 1, x);
				}
			}
		}
	\end{lstlisting}
	
	 时间复杂度:O(n²)  
	因为每次 'Insert(k, x)' 都要从头开始查找位置。
	
	
	
	\subsection*{练习24:链表转数组线性表}
	编写一个函数,把一个用单向链表表示的线性表转换成用数组表示的线性表。
	
	1)首先利用Chain的成员函数Find和LinearList的成员函数Insert来实现。该函数的时间复杂性是多少?试测试代码的正确性。
	
	2)利用链表遍历器来实现。函数的时间复杂性是多少?试用适当的测试数据来测试该函数
	的正确性。
	\subsubsection*{1) 使用 Find 和 Insert}
	
	\begin{lstlisting}[language=C++]
		template <typename T>
		void ChainToArray(Chain<T>& chain, LinearList<T>& list) {
			int pos = 1;
			ChainNode<T>* p = chain.First();
			while (p) {
				list.Insert(pos++, p->data);
				p = p->link;
			}
		}
	\end{lstlisting}
	
	 时间复杂度:O(n²)  
	同上,每次插入都需从头查找。
	
	\subsubsection*{2) 使用链表遍历器}
	
	\begin{lstlisting}[language=C++]
		template <typename T>
		void ChainToArrayFast(Chain<T>& chain, LinearList<T>& list) {
			ChainIterator<T> iter(chain);
			int pos = 1;
			while (T* x = iter.Next()) {
				list.Insert(pos++, *x);
			}
		}
	\end{lstlisting}
	
	 时间复杂度:O(n)  
	只需遍历一次链表即可。
	
	
	
	\subsection*{练习25:FromList / ToList 成员函数}
	扩充Chain的类定义,把LinearList转换成Chain以及Chain转换成LinearList的函数作为成员函数添加到Chain的类定义之中。具体任务是编写函数FromList(L)和ToList(L)。
	FromList(L)把一个线性表L转换成单向链表,而ToList(L)把一个单向链表转换成线性表L。每个函数的时间复杂性分别是多少?试测试代码的正确性。
	\begin{lstlisting}[language=C++]
		template <typename T>
		class Chain {
			public:
			void FromList(const LinearList<T>& L);
			void ToList(LinearList<T>& L) const;
		};
	\end{lstlisting}
	
	\begin{lstlisting}[language=C++]
		template <typename T>
		void Chain<T>::FromList(const LinearList<T>& L) {
			Clear();
			for (int i = 1; i <= L.Size(); ++i) {
				T x;
				if (L.Find(i, x))
				Insert(Size() + 1, x);
			}
		}
		
		template <typename T>
		void Chain<T>::ToList(LinearList<T>& L) const {
			L.Clear();
			int pos = 1;
			ChainNode<T>* p = first;
			while (p) {
				L.Insert(pos++, p->data);
				p = p->link;
			}
		}
	\end{lstlisting}
	
	 时间复杂度:O(n²) (FromList),O(n)(ToList)
	
	
	
	\subsection*{练习27:Reverse 反序链表}
	1)扩充Chain的类定义,增加函数Reverse,用于对x中的元素反序。要求反序操作就地进行,不需要分配任何新的节点。
	
	2)函数的时间复杂性是多少?
	
	3)通过编译和执行,测试函数的正确性。要求使用自己的测试数据。
	就地反序,不创建新节点。
	
	\begin{lstlisting}[language=C++]
		template <typename T>
		void Chain<T>::Reverse() {
			ChainNode<T>* prev = nullptr;
			ChainNode<T>* curr = first;
			while (curr) {
				ChainNode<T>* next = curr->link;
				curr->link = prev;
				prev = curr;
				curr = next;
			}
			std::swap(first, last);
		}
	\end{lstlisting}
	
	 时间复杂度:O(n)
	
	
	
	\subsection*{练习28:非成员函数 Reverse(A,B)}
	完成练习27,区别是Reverse不作为Chain的成员函数。要求利用Chain的成员函数来
	完成反序操作。新的函数将拥有两个参数A和B,A作为输入的链表,B是把A反序后得到的链表。在反序完成时,A变成一个空的链表。
	
	A 被清空,B 是 A 的反序。
	
	\begin{lstlisting}[language=C++]
		template <typename T>
		void Reverse(const Chain<T>& A, Chain<T>& B) {
			B.Clear();
			ChainNode<T>* p = A.first;
			while (p) {
				B.Insert(1, p->data); // 插入头部
				p = p->link;
			}
		}
	\end{lstlisting}
	
	 时间复杂度:O(n)
	
	
	
	\subsection*{练习29:Alternate 合并两个链表}
	令A和B都是Chain类型
	
	1)编写一个新的成员函数Alternate,用以创建一个新的线性表C,该表包含了A和B中的所有元素,其中A和B的元素轮流出现,表中的首元素为A中的第一个元素。在轮流排列元素时,如果某个表的元素用完了,则把另一个表的其余元素依次添加在新表的后部。代码的复杂性应与两个输入表的长度呈线性比例关系。
	
	2)证明代码具有线性复杂性。
	
	3)通过编译和执行,测试函数的正确性。要求使用自己的测试数据。
	
	轮流插入元素。
	
	\begin{lstlisting}[language=C++]
		template <typename T>
		void Chain<T>::Alternate(const Chain<T>& A, const Chain<T>& B) {
			Clear();
			ChainNode<T>* pa = A.first;
			ChainNode<T>* pb = B.first;
			
			while (pa || pb) {
				if (pa) { Insert(Size()+1, pa->data); pa = pa->link; }
				if (pb) { Insert(Size()+1, pb->data); pb = pb->link; }
			}
		}
	\end{lstlisting}
	
	 时间复杂度:O(m+n)
	
	
	
	\subsection*{练习30:破坏式 Alternate(A,B)}
	扩充Chain的类定义,增加一个与练习29中的Alternate函数相类似的函数Alternate。该
	函数应利用A和B中的物理节点来建立C。在执行完Alternate之后,A和B均变成空表。
	
	1)编写出Alternate的实现代码。代码的复杂性应与初始链表的长度呈线性关系。
	
	2)证明代码具有线性复杂性。
	
	3)通过编译和执行,测试函数的正确性。要求使用自己的测试数据。
	
	利用物理节点合并,A、B 清空。
	
	\begin{lstlisting}[language=C++]
		template <typename T>
		void Chain<T>::Alternate(Chain<T>& A, Chain<T>& B) {
			Clear();
			ChainNode<T>* pa = A.first;
			ChainNode<T>* pb = B.first;
			
			ChainNode<T>* tail = nullptr;
			while (pa || pb) {
				if (pa) {
					if (!first) first = tail = pa;
					else tail = tail->link = pa;
					pa = pa->link;
					tail->link = nullptr;
				}
				if (pb) {
					if (!first) first = tail = pb;
					else tail = tail->link = pb;
					pb = pb->link;
					tail->link = nullptr;
				}
			}
			A.first = A.last = nullptr;
			B.first = B.last = nullptr;
		}
	\end{lstlisting}
	
	 时间复杂度:O(m+n)
	
	
	
	\subsection*{练习31:Merge 合并两个有序链表}
	令A和B都是Chain类型,假定A和B的元素都是按序排列的(即从左至右按递增次
	序排列)
	
	1)编写一个函数Merge,用以创建一个新的有序线性表C,该表中包含了A和B的所有元素。
	
	2)函数的时间复杂性如何?
	
	3)通过编译和执行,测试函数的正确性。要求使用自己的测试数据。
	
	\begin{lstlisting}[language=C++]
		template <typename T>
		void Merge(const Chain<T>& A, const Chain<T>& B, Chain<T>& C) {
			ChainNode<T>* pa = A.first;
			ChainNode<T>* pb = B.first;
			
			while (pa && pb) {
				if (pa->data <= pb->data)
				C.Insert(C.Size()+1, pa->data), pa = pa->link;
				else
				C.Insert(C.Size()+1, pb->data), pb = pb->link;
			}
			while (pa) C.Insert(C.Size()+1, pa->data), pa = pa->link;
			while (pb) C.Insert(C.Size()+1, pb->data), pb = pb->link;
		}
	\end{lstlisting}
	
	 时间复杂度:O(m+n)
	
	
	
	\subsection*{练习33:Split 拆分奇偶位置元素}
	重做练习31,要求函数是Chain的一个成员函数,并使用两个输入链表中的物理节点来
	建立新链表C,在Merge执行完之后,两个输入链表均变成空表。
	
	\begin{lstlisting}[language=C++]
		template <typename T>
		void Split(const Chain<T>& C, Chain<T>& A, Chain<T>& B) {
			A.Clear(); B.Clear();
			ChainNode<T>* p = C.first;
			int index = 1;
			while (p) {
				if (index % 2 == 1)
				A.Insert(A.Size()+1, p->data);
				else
				B.Insert(B.Size()+1, p->data);
				p = p->link;
				index++;
			}
		}
	\end{lstlisting}
	
	 时间复杂度:O(n)
	
	
	
	\subsection*{练习41:循环链表删除节点x}
	令x指向一个循环链表z中的任意节点
	
	1)编写一个函数用来删除节点x中的元素。提示:由于不知道哪个节点是x的左边相邻节
	点,因此难以从链表中删除节点x;不过,为了删除x中的元素,可以把x的数据域用x的下一个节点y的数据域来替换,然后删除y节点即可。
	
	2)所编写的函数的时间复杂性如何?
	
	3)通过编译和执行,测试函数的正确性。要求使用自己的测试数据。
	
	思路:用下一个节点替换当前节点值,再删除下一个节点。
	
	\begin{lstlisting}[language=C++]
		template <typename T>
		void DeleteNode(ChainNode<T>* x) {
			if (x == x->link) {
				delete x;
				return;
			}
			ChainNode<T>* y = x->link;
			x->data = y->data;
			x->link = y->link;
			delete y;
		}
	\end{lstlisting}
	
	 时间复杂度:O(1)
	
	
	
	\subsection*{练习57:双向链表游标控制}
	为了有效地支持在一个双向链表中进行前移和后移,需要扩充Double的类定义(见程序3-21),即增加一个私有成员current,用它来纪录链表的当前位置,为此,需要增加以下共享函数:
	
	1)ResetLeft—将current置为LeftEnd。
	
	2)ResetRight—将current置为RightEnd。
	
	3)current(x)—取当前元素至x。如果current>length,函数返回false,否则返回true。
	
	4)End—如果当前的位置恰好指向链表的最后一个元素(即最右边的元素),则返回true,
	否则返回false。
	
	5)Front—如果当前的位置恰好指向链表的第一个元素(即最左边的元素),则返回true,
	否则返回false。
	
	6)Next—移动current指向链表的下一个元素。如果没有下一个元素,函数返回false,否
	则返回true。
	
	7)Previous—移动current指向链表的前一个元素。如果没有前一个元素,函数返回false,
	否则返回true。
	
	编写以上扩充函数,使用适当的测试数据测试代码的正确性。
	
	新增成员变量 'current',以及相关函数。
	
	\begin{lstlisting}[language=C++]
		template <typename T>
		class DoubleChain {
			public:
			void ResetLeft();
			void ResetRight();
			bool Current(T& x);
			bool End();
			bool Front();
			bool Next();
			bool Previous();
			
			private:
			DblNode<T>* current;
			DblNode<T>* head;
			int length;
		};
	\end{lstlisting}
	
	 时间复杂度:O(1)
	
	
	
	\subsection*{练习58:InsertionSort 插入排序链表}
	为Chain增加成员函数InsertionSort,该函数可利用程序2-15给出的插入排序算法对链表
	中的元素按递增次序进行重新排列。不得创建新的节点或删除老的节点。
	
	1)程序在最坏情况下的时间复杂性是多少?如果链表中的元素已经按递增次序排列,程序
	需要消耗多长时间?
	
	2)通过编译和执行,测试程序的正确性。要求使用自己的测试数据。
	
	\begin{lstlisting}[language=C++]
		template <typename T>
		void Chain<T>::InsertionSort() {
			if (!first || !first->link) return;
			
			ChainNode<T>* sorted = first;
			ChainNode<T>* curr = first->link;
			sorted->link = nullptr;
			
			while (curr) {
				ChainNode<T>* next = curr->link;
				if (curr->data <= sorted->data) {
					curr->link = sorted;
					sorted = curr;
				} else {
					ChainNode<T>* p = sorted;
					while (p->link && p->link->data < curr->data)
					p = p->link;
					curr->link = p->link;
					p->link = curr;
				}
				curr = next;
			}
			first = sorted;
		}
	\end{lstlisting}
	
	 时间复杂度:
	- 最坏情况 O(n²)
	- 已排序链表 O(n)
	
\section{3.5~~~寻址}

\subsection*{练习60:Output 成员函数和操作符 << 重载}
编写IndirectList的成员函数Output,并利用该函数重载操作符<<。测试代码的正确性。
\subsubsection*{实现代码}

\begin{lstlisting}[language=C++]
	template <typename T>
	class IndirectList {
		public:
		void Output(std::ostream& out) const;
		int length;
		
		private:
		T table; // 间接寻址表
	};
	
	template <typename T>
	void IndirectList<T>::Output(std::ostream& out) const {
		for (int i = 0; i < length; ++i)
		out << *table[i] << " ";
		out << std::endl;
	}
	
	// 重载 <<
	template <typename T>
	std::ostream& operator<<(std::ostream& out, const IndirectList<T>& list) {
		list.Output(out);
		return out;
	}
\end{lstlisting}

测试建议

\begin{lstlisting}[language=C++]
int main() {
	int a = 10, b = 20, c = 30;
	IndirectList<int> list;
	// 初始化 list.table[0] = &a 等
	std::cout << list;
	return 0;
}
\end{lstlisting}


\subsection*{练习61:Search 成员函数}
编写IndirectList的成员函数Search,测试其正确性,并指出函数的时间复杂性。

\begin{lstlisting}[language=C++]
	template <typename T>
	int IndirectList<T>::Search(const T& x) const {
		for (int i = 0; i < length; ++i)
		if (*table[i] == x)
		return i + 1; // 返回逻辑位置
		return -1;
	}
\end{lstlisting}

时间复杂度:O(n)


\subsection*{练习62:遍历器类 ChainIterator}
设计一个间接寻址的类遍历器,参照ChainIterator类(见程序3-18)的定义。通过用它
从左至右输出一个线性表来测试其正确性。

参照 `ChainIterator` 设计 `IndirectListIterator`。


\begin{lstlisting}[language=C++]
	template <typename T>
	class IndirectListIterator {
		public:
		IndirectListIterator(IndirectList<T>& list) : list(list), index(0) {}
		bool HasNext() { return index < list.length; }
		T& Next() { return *list.table[index++]; }
		
		private:
		IndirectList<T>& list;
		int index;
	};
\end{lstlisting}

测试建议

\begin{lstlisting}[language=C++]
IndirectList<int> list;
// 初始化
IndirectListIterator<int> iter(list);
while (iter.HasNext())
std::cout << iter.Next() << " ";
\end{lstlisting}


\subsection*{练习63:折半搜索函数 BinarySearch}
1)编写一个折半搜索(见程序2-30)函数,用来对一个间接寻址表进行搜索。假定对
于所有的$i\leq length-2$有$^*table[i]\leq ^*table[i+1]$。函数的时间复杂性应该是O(log(length)),试证
明之。

2)通过编译和执行,测试该程序的正确性。要求使用自己的测试数据。

3)搜索一个有序的单向链表速度会有多快?试编写一个具有这种时间复杂性的,对单向链
表进行搜索的函数。

适用于有序间接寻址表。


\begin{lstlisting}[language=C++]
	template <typename T>
	int BinarySearch(IndirectList<T>& list, const T& x) {
		int low = 0, high = list.length - 1;
		while (low <= high) {
			int mid = (low + high) / 2;
			if (*list.table[mid] == x) return mid + 1;
			else if (*list.table[mid] < x) low = mid + 1;
			else high = mid - 1;
		}
		return -1;
	}
\end{lstlisting}

时间复杂度:O(log n)

证明:
每次将搜索区间缩小一半,最多执行 $ \log_2 n $ 次循环。


\subsection*{练习64:插入排序函数 InsertionSort}
令x为IndirectList类型的对象

1)编写一个排序函数,使x按递增次序排列,即对于所有的$i\leq length-2$,有$^*table[i]\leq
table[i+l]$。基于插入排序方法(见程序2-15)实现该函数。函数的时间复杂性应该是
$O(length^2)$,并且与每个元素的大小无关。试证明之。

2)通过编译和执行,测试该程序的正确性。要求使用自己的测试数据。

对 IndirectList 进行插入排序。


\begin{lstlisting}[language=C++]
	template <typename T>
	void InsertionSort(IndirectList<T>& list) {
		for (int i = 1; i < list.length; ++i) {
			T* key = list.table[i];
			int j = i - 1;
			while (j >= 0 && *list.table[j] > *key) {
				list.table[j + 1] = list.table[j];
				--j;
			}
			list.table[j + 1] = key;
		}
	}
\end{lstlisting}

时间复杂度:O(n²)  
最坏情况下,每轮都要比较并移动前面所有元素。


\subsection*{练习65:冒泡/选择/计数排序实现}
分别采用以下排序算法(详见第2章)完成练习64:

1)冒泡排序。

2)选择排序。

3)计数排序。

\subsubsection*{1) 冒泡排序}

\begin{lstlisting}[language=C++]
	template <typename T>
	void BubbleSort(IndirectList<T>& list) {
		for (int i = 0; i < list.length - 1; ++i)
		for (int j = 0; j < list.length - i - 1; ++j)
		if (*list.table[j] > *list.table[j + 1])
		std::swap(list.table[j], list.table[j + 1]);
	}
\end{lstlisting}

时间复杂度:O(n²)

\subsubsection*{2) 选择排序}

\begin{lstlisting}[language=C++]
	template <typename T>
	void SelectionSort(IndirectList<T>& list) {
		for (int i = 0; i < list.length - 1; ++i) {
			int minIndex = i;
			for (int j = i + 1; j < list.length; ++j)
			if (*list.table[j] < *list.table[minIndex])
			minIndex = j;
			std::swap(list.table[i], list.table[minIndex]);
		}
	}
\end{lstlisting}

时间复杂度:O(n²)

\subsubsection*{3) 计数排序}

\begin{lstlisting}[language=C++]
	template <typename T>
	void CountingSort(IndirectList<T>& list, T maxVal) {
		int countSize = maxVal + 1;
		int* count = new int[countSize]{0};
		
		for (int i = 0; i < list.length; ++i)
		count[*list.table[i]]++;
		
		int index = 0;
		for (T val = 0; val <= maxVal; ++val)
		while (count[val]--)
		*list.table[index++] = val;
		
		delete[] count;
	}
\end{lstlisting}

时间复杂度:O(n + k),其中 k 是最大值

\subsection*{练习66:根据 table 数组排序 element 数组}
给定一个类型为T的数组element[0:length-l]和一个整数数组table[0:length-l]。table[]是
[0,1,...,length-1]的一种排列,使得对于$0\leq i\leq length-2$有$element[table[i]]\leq element[table
[i+l]]$。

1)编写一个函数对element[]进行排序,使得对于所有的i,有$element[i]\leq element[i+l]$。函
数的时间复杂性应该是O(s*length),其中s为每个元素的大小,函数的空间复杂性应该是O(s)。
试证明之。

2)测试函数的正确性。

\begin{lstlisting}[language=C++]
	template <typename T>
	void SortByTable(T element[], int table[], int length) {
		T* temp = new T[length];
		for (int i = 0; i < length; ++i)
		temp[i] = element[table[i]];
		for (int i = 0; i < length; ++i)
		element[i] = temp[i];
		delete[] temp;
	}
\end{lstlisting}

时间复杂度:O(n)  

空间复杂度:O(s),s 为每个元素大小

证明:
只需一次复制和一次赋值即可完成排序。
\section{3.6~~~}
\subsection*{练习67:为 SimChain 类设计遍历器 SimIterator}
为SimChain类设计一个类遍历器SimIterator。请参考程序3-18中关于Chains类遍历器的定义。SimIterator中应包含与ChainIterator相同的成员函数。编写并测试代码。

参考程序 3-18 中 ChainIterator 的定义。

\subsubsection*{SimIterator 实现}
\begin{lstlisting}[language=C++]
	template <typename T>
	class SimChain;
	
	template <typename T>
	class SimIterator {
		public:
		SimIterator(SimChain<T>& list) : list(list), current(0) {}
		
		bool HasNext() const;
		T& Next();
		
		private:
		SimChain<T>& list;
		int current;
	};
	
	template <typename T>
	bool SimIterator<T>::HasNext() const {
		return current < list.Length();
	}
	
	template <typename T>
	T& SimIterator<T>::Next() {
		if (!HasNext())
		throw std::out_of_range("No more elements");
		return *list[current++];
	}
\end{lstlisting}

\subsubsection*{测试建议}

\begin{lstlisting}[language=C++]
	SimChain<int> chain;
	chain.Insert(1, 10);
	chain.Insert(2, 20);
	SimIterator<int> iter(chain);
	while (iter.HasNext())
	std::cout << iter.Next() << " ";
\end{lstlisting}


\subsection*{练习68:修改 SimSpace 类以返回节点指针而非索引}
1)修改SimSpace类,使得Allocate返回一个指向node[i]的指针,而不是返回索引值i。类似地,修改Deallocate函数,使它的输入参数为欲释放节点的指针。

2)采用1)中的SimSpace代码重新编写SimChain的实现代码。请留意新代码与Chain的代码之间的相似性。

\subsubsection*{1) 修改 Allocate / Deallocate 函数}

\begin{lstlisting}[language=C++]
	template <typename T>
	class SimSpace {
		public:
		T* Allocate();
		void Deallocate(T* ptr);
		
		private:
		int freeList; // 空闲节点索引
		T* nodes;
		int capacity;
	};
\end{lstlisting}

\begin{lstlisting}[language=C++]
	template <typename T>
	T* SimSpace<T>::Allocate() {
		if (freeList == -1)
		throw std::bad_alloc();
		int index = freeList;
		freeList = *(int*)&nodes[index]; // 获取下一个空闲节点
		return &nodes[index];
	}
	
	template <typename T>
	void SimSpace<T>::Deallocate(T* ptr) {
		*(int*)ptr = freeList;
		freeList = ptr - nodes;
	}
\end{lstlisting}

\subsubsection*{2) 修改 SimChain 使用新接口}

\begin{lstlisting}[language=C++]
	SimSpace<Node<T>> space;
	Node<T>* node = space.Allocate();
	space.Deallocate(node);
\end{lstlisting}

时间复杂度:O(1)


\subsection*{练习69:使用静态 SimSpace 成员优化内存分配}
1)修改SimNode类的定义,为它添加一个类型为SimSpace<T>的静态成员S。这样,所有类型为SimSpace<T>的节点都可以共享同样的模拟空间。重载函数new和delete,以便从模拟空间S中取得节点SimNodes或将SimNodes节点送回模拟空间S中。

2)假定SimSpace是按照练习68的要求实现的,SimNode是按照1)中要求实现的。修改Chain类的代码(见程序3-8),使得它能够用SimNodes代替ChainNodes进行工作。测试代码,并测量运行时间,以确定哪个版本的Chain更快。
\subsubsection*{1) 修改 SimNode 类,添加静态 SimSpace 成员}

\begin{lstlisting}[language=C++]
	template <typename T>
	class SimNode {
		public:
		static SimSpace<SimNode<T>> space;
		
		void* operator new(size_t size) {
			return space.Allocate();
		}
		
		void operator delete(void* ptr) {
			space.Deallocate(static_cast<SimNode<T>*>(ptr));
		}
		
		T data;
		SimNode<T>* link;
	};
\end{lstlisting}

\begin{lstlisting}[language=C++]
	// 静态成员初始化
	template <typename T>
	SimSpace<SimNode<T>> SimNode<T>::space;
\end{lstlisting}

\subsubsection*{2) 修改 Chain 类使用 SimNode 替代 ChainNode}

\begin{lstlisting}[language=C++]
	typedef SimNode<T> Node;
	Node* first;
\end{lstlisting}

性能对比建议

分别运行以下操作各 10000 次:
- 插入 1000 个节点
- 删除 1000 个节点
记录平均耗时,比较原 Chain 和 SimChain 性能差异。


\subsection*{练习70:插入排序链表}
假定一个链表是采用模拟指针进行描述的,节点的类型为SimNode

1)编写一个程序,该程序使用插入排序算法对链表中的节点进行重新排序,要求按照data域的递增次序进行排列。

2)代码的时间复杂性是多少?如果不是O(n2),请重写代码以使其具有这样的复杂性,其中n为链表长度。

3)测试代码的正确性。
\subsubsection*{1) 插入排序函数}
\begin{lstlisting}[language=C++]
	template <typename T>
	void InsertionSort(SimChain<T>& chain) {
		if (chain.IsEmpty()) return;
		
		SimNode<T>* sorted = chain.First();
		SimNode<T>* curr = chain.First()->link;
		
		while (curr) {
			SimNode<T>* next = curr->link;
			if (curr->data <= sorted->data) {
				curr->link = sorted;
				sorted = curr;
			} else {
				SimNode<T>* p = sorted;
				while (p->link && p->link->data < curr->data)
				p = p->link;
				curr->link = p->link;
				p->link = curr;
			}
			curr = next;
		}
		chain.SetFirst(sorted);
	}
\end{lstlisting}

\subsubsection*{2) 时间复杂度分析}

最坏情况每次都要扫描整个有序部分:

时间复杂度:O(n²)


\subsection*{练习74:自定义 new/delete 提升性能}
对new和delete的调用通常都要耗费很多时间,为此,可以使用自行编写的释放函数(该函数能将被删除的节点放入自由节点表中)来替换delete函数,以便提高代码的运行效率。为了替换new,可以自行编写一个分配函数,每当自由节点链表为空时该函数才会去调用new。修改Chain类(见程序3-8)以实现上述思想。要求编写如上所述的分配节点和释放节点函数,并对自由节点链表进行初始化。试比较两种Chain版本的执行时间,并评价新方法的优缺点。
\subsubsection*{1) 自定义分配/释放函数}

\begin{lstlisting}[language=C++]
	template <typename T>
	class FreeList {
		public:
		static void* operator new(size_t size);
		static void operator delete(void* ptr);
		
		private:
		static std::stack<T*> freeNodes;
	};
	
	template <typename T>
	std::stack<T*> FreeList<T>::freeNodes;
	
	template <typename T>
	void* FreeList<T>::operator new(size_t size) {
		if (!freeNodes.empty()) {
			T* node = freeNodes.top();
			freeNodes.pop();
			return node;
		}
		return ::operator new(size);
	}
	
	template <typename T>
	void FreeList<T>::operator delete(void* ptr) {
		freeNodes.push(static_cast<T*>(ptr));
	}
\end{lstlisting}

\subsubsection*{2) 修改 Chain 类使用自定义分配器}

\begin{lstlisting}[language=C++]
	typedef FreeList<Node<T>> FastNode;
	FastNode<T>* node = new FastNode<T>;
	delete node;
\end{lstlisting}
\end{document}