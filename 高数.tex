\documentclass[UTF8]{ctexart}
\usepackage{geometry}
\usepackage{fontspec}
\usepackage{amsmath, amssymb, amsthm}
\usepackage{xcolor}
\usepackage{multicol}
\usepackage{tikz}
\usetikzlibrary{shapes.geometric}

% 设置页面边距
\geometry{a4paper, margin=1cm}

% 设置字体
\setmainfont{Times New Roman}[Scale=1.0]
\newcommand{\simsun}{\CJKfamily{gbsn}\zihao{-5}} % 小五宋体

\title{考研高数积分公式 Cheat Sheet}
\author{}
\date{}

\begin{document}
	
	\maketitle
	
	\section*{一、基本积分公式(必背)}
	\begin{multicols}{2}
		
		
		- $\displaystyle \int x^n dx = \frac{x^{n+1}}{n+1} + C \quad (n \neq -1)$
		
		- $\displaystyle \int \frac{1}{x} dx = \ln|x| + C$
		
		- $\displaystyle \int e^x dx = e^x + C$
		
		- $\displaystyle \int a^x dx = \frac{a^x}{\ln a} + C$
		
		- $\displaystyle \int \sin x dx = -\cos x + C$
		
		- $\displaystyle \int \cos x dx = \sin x + C$
		
		- $\displaystyle \int \sec^2 x dx = \tan x + C$
		
		- $\displaystyle \int \csc^2 x dx = -\cot x + C$
		
		- $\displaystyle \int \sec x \tan x dx = \sec x + C$
		
		- $\displaystyle \int \csc x \cot x dx = -\csc x + C$
		
		- $\displaystyle \int \frac{1}{1+x^2} dx = \arctan x + C$
		
		- $\displaystyle \int \frac{1}{\sqrt{1-x^2}} dx = \arcsin x + C$
		
	\end{multicols}
	
	\section*{二、积分技巧与方法}
	
	
	- \textbf{换元积分法}:
	$$
	\int f(g(x))g'(x)dx = \int f(u)du \quad \text{其中 } u = g(x)
	$$
	
	- \textbf{分部积分法}:
	$$
	\int u dv = uv - \int v du
	$$
	助记口诀:“\textbf{反导对}”,即:
	\begin{center}
		\textit{反函数 → 对数函数 → 幂函数 → 三角函数 → 指数函数}
	\end{center}
	
	- \textbf{三角代换}:
	$$
	\sqrt{a^2 - x^2} \Rightarrow x = a\sin\theta,\quad
	\sqrt{a^2 + x^2} \Rightarrow x = a\tan\theta,\quad
	\sqrt{x^2 - a^2} \Rightarrow x = a\sec\theta
	$$
	
	
	\section*{三、定积分重要性质}
	
	
	- $\displaystyle \int_a^b f(x)dx = -\int_b^a f(x)dx$
	
	- $\displaystyle \int_a^a f(x)dx = 0$
	
	- $\displaystyle \int_a^b [f(x) \pm g(x)]dx = \int_a^b f(x)dx \pm \int_a^b g(x)dx$
	
	- $\displaystyle \int_a^b k f(x)dx = k \int_a^b f(x)dx$
	
	- $\displaystyle \int_a^b f(x)dx = \int_a^c f(x)dx + \int_c^b f(x)dx$
	
	
	\section*{四、微积分基本定理}
	$$
	F(x) = \int_a^x f(t)dt \Rightarrow F'(x) = f(x)
	$$
	$$
	\int_a^b f(x)dx = F(b) - F(a)
	$$
	
	\section*{五、广义积分}
	$$
	\int_a^\infty f(x)dx = \lim_{b \to \infty} \int_a^b f(x)dx
	$$
	$$
	\int_{-\infty}^b f(x)dx = \lim_{a \to -\infty} \int_a^b f(x)dx
	$$
	
	\section*{六、常用积分表}
	\begin{tabular}{|l|l|}
		\hline
		$\int \tan x dx$ & $-\ln|\cos x| + C$ \\
		$\int \cot x dx$ & $\ln|\sin x| + C$ \\
		$\int \sec x dx$ & $\ln|\sec x + \tan x| + C$ \\
		$\int \csc x dx$ & $-\ln|\csc x + \cot x| + C$ \\
		\hline
	\end{tabular}
	
	\section*{七、Flashcards 记忆卡模板(可打印)}
	\begin{tikzpicture}
		\foreach \i in {0,1,...,3} {
			\draw (\i*5,0) rectangle (\i*5+4.5,6);
			\node at (\i*5+2.25,5) {\simsun 示例公式};
			\node at (\i*5+2.25,3) {\tiny $\displaystyle \int \sin x dx = -\cos x + C$};
		}
		\foreach \i in {0,1,...,3} {
			\draw (\i*5,7) rectangle (\i*5+4.5,13);
			\node at (\i*5+2.25,12) {\simsun 分部积分口诀};
			\node at (\i*5+2.25,10) {\Large 反导对};
		}
	\end{tikzpicture}
	
	\section*{八、Mnemonics(助记法)}
	
	
	- \textbf{分部积分顺序口诀}:
	\begin{center}
		“\textbf{反导对}” —— 反函数优先作为 $u$,其次是导数简单函数,最后是对数函数。
	\end{center}
	
	- \textbf{三角代换口诀}:
	\begin{center}
		“\textbf{正弦平方减,正切平方加,正割差平方}”
	\end{center}
	即:
	\begin{align*}
		\sqrt{a^2 - x^2} &\Rightarrow x = a\sin\theta \\
		\sqrt{a^2 + x^2} &\Rightarrow x = a\tan\theta \\
		\sqrt{x^2 - a^2} &\Rightarrow x = a\sec\theta
	\end{align*}
	
	
\end{document}