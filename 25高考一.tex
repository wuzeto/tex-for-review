\documentclass[UTF8]{ctexart}
\usepackage{amsmath, amssymb}
\usepackage{enumitem}
\usepackage{tasks}
\usepackage{graphicx}
\usepackage[a4paper, margin=0.6in]{geometry}
\usepackage{amsmath}
\usepackage{amssymb}

\begin{document}
	\title{2025年普通高等学校招生全国统一考试 数学}
	\author{}
	\date{}
	\maketitle
	
	\begin{center}
		\begin{minipage}{\linewidth}
			\textbf{注意事项:}
			\begin{enumerate}
				\item 答题前,务必将自己的姓名、准考证号用0.5毫米黑色墨水的签字笔填写在试卷及答题卡的规定位置。
				\item 请认真核对监考员在答题卡上所粘贴的条形码上的姓名、准考证号与本人是否相符。
				\item 作答选择题必须用2B铅笔将答题卡上对应选项的方框涂满、涂黑;如需改动,请用橡皮擦干净后,再选涂其他答案。作答非选择题,必须用0.5毫米黑色墨水的签字笔在答题卡上的指定位置作答,在其他位置作答一律无效。
				\item 本试卷共4页,满分150分,考试时间为120分钟。考试结束后,请将本试卷和答题卡一并交回。
			\end{enumerate}
		\end{minipage}
	\end{center}
	
	\section*{一、选择题}
	本大题共8小题,每小题5分,共计40分. 每小题给出的四个选项中,只有一个选项是正确的. 请把正确的选项填涂在答题卡相应的位置上.
	
	\begin{enumerate}
		\item \((1 + 5i)i\)的虚部为
		\begin{enumerate}[label = A.]
			\item -1
			\item 0
			\item 1
			\item 6
		\end{enumerate}
		\item 设全集\(U = \{1,2,3,4,5,6,7,8,9\}\),集合\(A = \{1,3,5\}\),则\(\complement_U A\)中元素个数为
		\begin{enumerate}[label = A.]
			\item 0
			\item 3
			\item 5
			\item 8
		\end{enumerate}
		\item 若双曲线\(C\)的虚轴长为实轴长的7倍,则\(C\)的离心率为
		\begin{enumerate}[label = A.]
			\item \(\sqrt{2}\)
			\item 2
			\item \(\sqrt{7}\)
			\item \(2\sqrt{2}\)
		\end{enumerate}
		\item 若点\((a,0)(a > 0)\)是函数\(y = 2\tan(x - \frac{\pi}{3})\)的图象的一个对称中心,则\(a\)的最小值为
		\begin{enumerate}[label = A.]
			\item \(30^{\circ}\)
			\item \(60^{\circ}\)
			\item \(90^{\circ}\)
			\item \(135^{\circ}\)
		\end{enumerate}
		\item 设\(f(x)\)是定义在\(\mathbb{R}\)上且周期为2的偶函数,当\(2\leq x\leq3\)时,\(f(x)=5 - 2x\),则\(f(-\frac{3}{4})=\)
		\begin{enumerate}[label = A.]
			\item \(-\frac{1}{2}\)
			\item \(-\frac{1}{4}\)
			\item \(\frac{1}{4}\)
			\item \(\frac{1}{2}\)
		\end{enumerate}
		\item 帆船运动员借助风力驾驶帆船,
		\item 若圆\(x^{2}+(y + 2)^{2}=r^{2}(r > 0)\)上到直线\(y=\sqrt{3}x + 2\)的距离为1的点有且仅有2个,则\(r\)的取值范围是
		\begin{enumerate}[label = A.]
			\item \((0,1)\)
			\item \((1,3)\)
			\item \((3,+\infty)\)
			\item \((0,+\infty)\)
		\end{enumerate}
		\item 若实数\(x\),\(y\),\(z\)满足\(2+\log_2 x = 3+\log_3 y = 5+\log_5 z\),则\(x\),\(y\),\(z\)的大小关系不可能是
		\begin{enumerate}[label = A.]
			\item \(x > y > z\)
			\item \(x > z > y\)
			\item \(y > x > z\)
			\item \(y > z > x\)
		\end{enumerate}
	\end{enumerate}
	
\end{document}