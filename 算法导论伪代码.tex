\documentclass[UTF8]{ctexart}
\usepackage{enumitem}
\usepackage{titlesec}
\usepackage{amsmath, amssymb}
\usepackage{enumitem}
\usepackage{geometry}
\usepackage{algorithm}
\usepackage{algorithmic}
\usepackage{listings}
\usepackage{fancyhdr}
\usepackage{graphicx}
\usepackage{amsmath}
\usepackage{xcolor}

% 自定义样式
\lstset{
	basicstyle=\ttfamily,
	keywordstyle=\color{blue},
	stringstyle=\color{magenta},
	commentstyle=\color{gray},
	showstringspaces=false,
	frame=single,
	breaklines=true,
	numbers=none,
	numberstyle=\tiny\color{gray},
	stepnumber=1,
	tabsize=4,
	captionpos=b,
	escapeinside={(*@}{@*)},
}

% 设置页眉
\pagestyle{fancy}
\fancyhf{}
\rhead{\textbf{ADT 描述}}
\lhead{\textbf{线性表、数组、栈、队列}}


\begin{document}
	
	\begin{algorithm}
		\caption{Huffman Algorithm (HUFFMAN(C))}
		\label{alg:huffman}
		\begin{algorithmic}[1]
			\STATE $ n = |C| $
			\STATE $ Q = C $
			\FOR{$ i = 1 $ \TO $ n - 1 $}
			\STATE allocate a new node $ z $
			\STATE $ z.\text{left} = x = \text{EXTRACT-MIN}(Q) $
			\STATE $ z.\text{right} = y = \text{EXTRACT-MIN}(Q) $
			\STATE $ z.\text{freq} = x.\text{freq} + y.\text{freq} $
			\STATE $\text{INSERT}(Q, z)$
			\ENDFOR
			\RETURN $\text{EXTRACT-MIN}(Q)$
		\end{algorithmic}
	\end{algorithm}
	
	\begin{algorithm}
		\caption{B-TREE-SEARCH(x, k)}
		\label{alg:btree_search}
		\begin{algorithmic}[1]
			\STATE $ i \gets 1 $
			\WHILE{$ i \leq x.n $ and $ k > x.key_i $}
			\STATE $ i \gets i + 1 $
			\ENDWHILE
			\IF{$ i \leq x.n $ and $ k == x.key_i $}
			\RETURN $ (x, i) $
			\ELSIF{ $ x.leaf $ }
			\RETURN NIL
			\ELSE
			\STATE DISK-READ($ x.c_i $)
			\RETURN B-TREE-SEARCH($ x.c_i, k $)
			\ENDIF
		\end{algorithmic}
	\end{algorithm}
	
	\section{B-TREE-CREATE}
	\begin{algorithm}
		\caption{B-TREE-CREATE(x, k)}
		\label{alg:btree_search}
		\begin{algorithmic}[1]

			\STATE $ x.\text{c}_{i + 1} \gets z $
			\FOR{$ j = x.n $ downto $ i $}
			\STATE $ x.\text{key}_{j + 1} \gets x.\text{key}_j $
			\ENDFOR
			\STATE $ x.\text{key}_i \gets y.\text{key}_t $
			\STATE $ x.n \gets x.n + 1 $
			\STATE $\text{DISK-WRITE}(y)$
			\STATE $\text{DISK-WRITE}(z)$
			\STATE $\text{DISK-WRITE}(x)$
	\end{algorithmic}
\end{algorithm}

\section{B-TREE-INSERT-NONFULL}

\begin{algorithm}
\caption{B-TREE-INSERT-NONFULL(x, k)}
\label{alg:btree_insert_nonfull}
\begin{algorithmic}[1]
\STATE $ i \gets x.n $
\IF{$ x.\text{leaf} $}
\WHILE{$ i \geq 1 $ and $ k < x.\text{key}_i $}
\STATE $ x.\text{key}_{i + 1} \gets x.\text{key}_i $
\STATE $ i \gets i - 1 $
\ENDWHILE
\STATE $ x.\text{key}_{i + 1} \gets k $
\STATE $ x.n \gets x.n + 1 $
\STATE $\text{DISK-WRITE}(x)$
\ELSIF{$ i \geq 1 $ and $ k < x.\text{key}_i $}
\STATE $ i \gets i - 1 $
\STATE $ i \gets i + 1 $
\STATE $\text{DISK-READ}(x.\text{c}_i)$
\IF{$ x.\text{c}_i.n == 2t - 1 $}
\STATE $\text{B-TREE-SPLIT-CHILD}(x, i)$
\IF{$ k > x.\text{key}_i $}
\STATE $ i \gets i + 1 $
\ENDIF
\ENDIF
\STATE $\text{B-TREE-INSERT-NONFULL}(x.\text{c}_i, k)$
\ENDIF
\end{algorithmic}
\end{algorithm}
	
		
	\section*{ADT 3-1 线性表的抽象数据类型描述}
	
	\subsection*{抽象数据类型 LinearList \{}
	
	\subsubsection*{实例}
	0 或多个元素的有序集合
	
	\subsubsection*{操作}
	\begin{lstlisting}[language=Python, caption={LinearList ADT}]
		Create (): 创建一个空线性表
		Destroy (): 删除表
		IsEmpty(): 如果表为空则返回 true,否则返回 false
		Length (): 返回表的大小 (即表中元素个数)
		Find (k,x): 寻找表中第 k 个元素,并把它保存到 x 中;如果不存在,则返回 false
		Search (x): 返回元素 x 在表中的位置;如果 x 不在表中,则返回 0
		Delete (k,x): 删除表中第 k 个元素,并把它保存到 x 中;函数返回修改后的线性表
		Insert (k,x): 在第 k 个元素之后插入 x;函数返回修改后的线性表
		Output (out): 把线性表放入输出流 out 之中
	\end{lstlisting}
	
	\section*{ADT 4-1 数组的抽象数据类型描述}
	
	\subsection*{抽象数据类型 Array \{}
	
	\subsubsection*{实例}
	形如 (index, value) 的数据对集合,其中任意两对数据的 index 值都各不相同
	
	\subsubsection*{操作}
	\begin{lstlisting}[language=Python, caption={Array ADT}]
		Create(): 创建一个空的数组
		Store(index, value): 添加数据 (index, value),同时删除具有相同 index 值的数据对(如果存在)
		Retrieve(index): 返回索引值为 index 的数据对
	\end{lstlisting}
	
	\section*{ADT 5-1 堆的抽象数据类型描述}
	
	\subsection*{抽象数据类型 Stack \{}
	
	\subsubsection*{实例}
	元素线性表,栈底,栈顶
	
	\subsubsection*{操作}
	\begin{lstlisting}[language=Python, caption={Stack ADT}]
		Create(): 创建一个空的堆栈
		IsEmpty(): 如果堆栈为空,则返回 true,否则返回 false
		IsFull(): 如果堆栈满,则返回 true,否则返回 false
		Top(): 返回栈顶元素
		Add(x): 向堆栈中添加元素 x
		Delete(x): 删除栈顶元素,并将它传递给 x
	\end{lstlisting}
	
	\section*{ADT 6-1 队列的抽象数据类型描述}
	
	\subsection*{抽象数据类型 Queue \{}
	
	\subsubsection*{实例}
	有序线性表,一端称为 front,另一端称为 rear
	
	\subsubsection*{操作}
	\begin{lstlisting}[language=Python, caption={Queue ADT}]
		Create (): 创建一个空的队列;
		IsEmpty (): 如果队列为空,则返回 true,否则返回 false;
		IsFull (): 如果队列满,则返回 true;否则返回 false;
		First (): 返回队列的第一个元素;
		Last (): 返回队列的最后一个元素;
		Add (x): 向队列中添加元素 x;
		Delete (x): 删除队首元素,并送入 x;
	\end{lstlisting}
	
\end{document}