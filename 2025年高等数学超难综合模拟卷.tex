% Options for packages loaded elsewhere
\PassOptionsToPackage{unicode}{hyperref}
\PassOptionsToPackage{hyphens}{url}
\documentclass[UTF8]{ctexart}
\usepackage{xcolor}
\usepackage{amsmath,amssymb}
\setcounter{secnumdepth}{-\maxdimen} % remove section numbering
\usepackage{iftex}
\ifPDFTeX
  \usepackage[T1]{fontenc}
  \usepackage[utf8]{inputenc}
  \usepackage{textcomp} % provide euro and other symbols
\else % if luatex or xetex
  \usepackage{unicode-math} % this also loads fontspec
  \defaultfontfeatures{Scale=MatchLowercase}
  \defaultfontfeatures[\rmfamily]{Ligatures=TeX,Scale=1}
\fi
\usepackage{lmodern}
\ifPDFTeX\else
  % xetex/luatex font selection
\fi
% Use upquote if available, for straight quotes in verbatim environments
\IfFileExists{upquote.sty}{\usepackage{upquote}}{}
\IfFileExists{microtype.sty}{% use microtype if available
  \usepackage[]{microtype}
  \UseMicrotypeSet[protrusion]{basicmath} % disable protrusion for tt fonts
}{}
\makeatletter
\@ifundefined{KOMAClassName}{% if non-KOMA class
  \IfFileExists{parskip.sty}{%
    \usepackage{parskip}
  }{% else
    \setlength{\parindent}{0pt}
    \setlength{\parskip}{6pt plus 2pt minus 1pt}}
}{% if KOMA class
  \KOMAoptions{parskip=half}}
\makeatother
\setlength{\emergencystretch}{3em} % prevent overfull lines
\providecommand{\tightlist}{%
  \setlength{\itemsep}{0pt}\setlength{\parskip}{0pt}}
\usepackage{bookmark}
\IfFileExists{xurl.sty}{\usepackage{xurl}}{} % add URL line breaks if available
\urlstyle{same}
\hypersetup{
  hidelinks,
  pdfcreator={LaTeX via pandoc}}
\usepackage[a4paper, margin=0.6in]{geometry}
\author{}
\date{}

\begin{document}

\section{2025年高等数学超难综合模拟卷}\label{2025ux5e74ux9ad8ux7b49ux6570ux5b66ux8d85ux96beux7efcux5408ux6a21ux62dfux5377}

\subsection{一、单项选择题(每题5分,共40分)}\label{ux4e00ux5355ux9879ux9009ux62e9ux9898ux6bcfux98985ux5206ux517140ux5206uxff09}

\begin{enumerate}
\def\labelenumi{\arabic{enumi}.}
\item
  设函数
  \( f(x) = \int_0^{x^2} \sin(\sqrt{t}) e^{-\frac{1}{1+t}} dt \),则
  \( \lim_{x \to 0} \frac{f(x) - x^3 \arctan x}{x^6} = \)\\
  (A) \( -\frac{1}{10} \) (B) \( \frac{1}{10} \) (C) \( -\frac{1}{12} \)
  (D) \( \frac{1}{12} \)
\item
  已知5阶矩阵 \( A \) 满足 \( A^3 = O \) 且
  \( \text{rank}(A^2) = 2 \),则 \( A \)
  的Jordan标准形中1阶Jordan块的数量为\\
  (A) 1 (B) 2 (C) 3 (D) 4
\item
  设二维随机变量 \( (X,Y) \) 满足 \( X \sim N(0,1) \),且在 \( X=x \)
  条件下 \( Y \sim N(x, x^2) \),则 \( \text{Cov}(X,Y) \) 和 \( D(Y) \)
  分别为\\
  (A) 1, 2 (B) 1, 3 (C) 2, 3 (D) 2, 4
\item
  微分方程 \( (x^2 y''')^3 + e^{y'} = x \ln x \)
  的阶数与是否为线性方程的判断为\\
  (A) 3阶,非线性 (B) 3阶,线性 (C) 2阶,非线性 (D) 2阶,线性
\item
  设 \( \Omega \) 由曲面 \( z = \sqrt{x^2 + y^2} \) 与
  \( z = \sqrt{2 - x^2 - y^2} \) 围成,计算
  \( \iiint_{\Omega} (x^4 + y^4 + z^4) dV \)
  时,利用对称性化简后积分可表示为\\
  (A) \( 4 \iiint_{\Omega} z^4 dV \) (B)
  \( \frac{4}{3} \iiint_{\Omega} (x^4 + y^4 + z^4) dV \)\\
  (C) \( \frac{2}{3} \iiint_{\Omega} (x^4 + y^4 + z^4) dV \) (D)
  \( \frac{8}{3} \iiint_{\Omega} z^4 dV \)
\item
  级数 \( \sum_{n=1}^{\infty} \frac{(-1)^n n!}{n^n} (2+\sin n) \)
  的敛散性为\\
  (A) 绝对收敛 (B) 条件收敛 (C) 发散 (D) 无法判断
\item
  设向量组 \( \alpha_1, \alpha_2, \alpha_3 \) 线性无关,向量组
  \( \beta_1 = \alpha_1 + k\alpha_2 \),
  \( \beta_2 = \alpha_2 + k\alpha_3 \),
  \( \beta_3 = \alpha_3 + k\alpha_1 \),当 \( k = \) 何值时
  \( \beta_1, \beta_2, \beta_3 \) 线性相关?\\
  (A) 1 (B) -1 (C) 2 (D) -2
\item
  设总体 \( X \sim U(0, \theta) \),\( X_1, X_2, X_3 \)
  为样本,下列估计量中最有效的 \( \theta \) 估计量是\\
  (A) \( \hat{\theta}_1 = 2\overline{X} \) (B)
  \( \hat{\theta}_2 = \frac{4}{3} \max\{X_1,X_2,X_3\} \)\\
  (C) \( \hat{\theta}_3 = 3\min\{X_1,X_2,X_3\} \) (D)
  \( \hat{\theta}_4 = \overline{X} + \max\{X_1,X_2,X_3\} \)
\end{enumerate}

\subsection{二、填空题(每题5分,共30分)}\label{ux4e8cux586bux7a7aux9898ux6bcfux98985ux5206ux517130ux5206uxff09}

\begin{enumerate}
\def\labelenumi{\arabic{enumi}.}
\item
  求极限
  \( \lim_{n \to \infty} n^2 \left(1 - \cos\frac{1}{n} - \int_0^{\frac{1}{n}} \arctan t dt\right) = \)
  \textbf{\_\_}
\item
  设矩阵 \( A = \begin{pmatrix} 
  1 & 1 & 1 \\
  1 & 1 & 1 \\
  1 & 1 & 1 
  \end{pmatrix} \),则二次型 \( f = x^T A^5 x \) 的规范形为
  \textbf{\_\_}
\item
  已知随机变量 \( X \) 满足 \( E(X) = 1 \), \( E(X^2) = 4 \),
  \( E(X^3) = 10 \),则 \( X \) 的三阶中心矩为 \textbf{\_\_}
\item
  微分方程 \( y''' - 3y'' + 3y' - y = e^x \sqrt{x} \) 的特解形式应设为
  \textbf{\_\_}
\item
  幂级数 \( \sum_{n=1}^{\infty} \frac{(n!)^2}{(2n)!} (x-1)^n \)
  的收敛半径为 \textbf{\_\_}
\item
  设 \( f(x,y,z) = x^2 + y^2 + z^2 \),则 \( f \) 在点 \( (1,1,1) \)
  处沿曲面 \( x^2 + y^2 + z^2 = 3 \) 外法线方向的方向导数为
  \textbf{\_\_}
\end{enumerate}

\subsection{三、解答题(共130分)}\label{ux4e09ux89e3ux7b54ux9898ux5171130ux5206uxff09}

15.(12分)设函数
\( f(x) = \int_0^x e^{-t^2} dt \cdot \int_0^x t e^{-t^2} dt \),\\
(i) 求 \( f(x) \) 的n阶导数 \( f^{(n)}(0) \);\\
(ii) 求 \( f(x) \) 的麦克劳林级数展开式(前3个非零项)。

16.(12分)已知矩阵 \( A = \begin{pmatrix} 
1 & -1 & 1 & 0 \\
2 & 4 & -2 & 1 \\
-3 & -3 & 5 & 0 \\
0 & 0 & 0 & 2 
\end{pmatrix} \),\\
(i) 求 \( A \) 的特征值和特征向量;\\
(ii) 求 \( e^A \) 的矩阵表达式。

17.(12分)设二维随机变量 \( (X,Y) \) 的联合概率密度为

\[f(x,y) = \begin{cases} 
k(x^2 + y^2), & x^2 + y^2 \leq 1 \\
0, & 其他 
\end{cases}\]

(i) 求常数 \( k \);\\
(ii) 求 \( Z = X^2 + Y^2 \) 的概率密度;\\
(iii) 求 \( E(\sqrt{X^2 + Y^2}) \)。

18.(12分)求微分方程 \( y'' + y = \frac{1}{\cos x} \)
的通解,并计算满足初始条件 \( y(0)=1, y'(0)=0 \) 的特解。

19.(14分)计算曲面积分 \( \iint_{\Sigma} (x^3 + y^3 + z^3) dS \),其中
\( \Sigma \) 为曲面 \( z = \sqrt{a^2 - x^2 - y^2} \) 与 \( z = 0 \)
所围立体的表面。

20.(14分)设级数 \( \sum_{n=1}^{\infty} a_n \) 收敛,且
\( a_n > 0 \),证明:\\
(i) \( \sum_{n=1}^{\infty} \frac{\sqrt{a_n}}{n} \) 收敛;\\
(ii) \( \sum_{n=1}^{\infty} a_n^{1 - \frac{1}{n}} \) 收敛当且仅当
\( \sum_{n=1}^{\infty} a_n \) 绝对收敛。

21.(14分)设向量空间 \( V \) 是所有2×2实矩阵构成的空间,定义线性变换
\( T: V \to V \) 为 \( T(A) = A + A^T \),\\
(i) 求 \( T \) 的核 \( \ker(T) \) 和像 \( \text{im}(T) \);\\
(ii) 求 \( T \) 在基 \( E_{11}, E_{12}, E_{21}, E_{22} \)
下的矩阵表示;\\
(iii) 证明 \( V = \ker(T) \oplus \text{im}(T) \)。

22.(14分)设总体 \( X \) 的概率密度为

\[f(x;\theta) = \begin{cases} 
\frac{1}{\theta} e^{-\frac{x-\mu}{\theta}}, & x \geq \mu \\
0, & x < \mu 
\end{cases}\]

其中 \( \theta > 0 \) 和 \( \mu \)
为未知参数,\( X_1, X_2, \dots, X_n \) 为样本,\\
(i) 求 \( \theta \) 和 \( \mu \) 的矩估计量;\\
(ii) 求 \( \theta \) 和 \( \mu \) 的极大似然估计量;\\
(iii) 若 \( \mu = 0 \),构造 \( \theta \) 的一个无偏估计量。

23.(16分)设函数 \( f(x,y) = \frac{1}{1 - xy} \),\\
(i) 求 \( f(x,y) \) 在点 \( (0,0) \) 处的n阶偏导数
\( \frac{\partial^n f(0,0)}{\partial x^k \partial y^{n-k}} \);\\
(ii) 证明 \( f(x,y) \) 在 \( |x| < 1, |y| < 1 \) 内可微;\\
(iii) 计算二重积分 \( \iint_{D} \frac{1}{(1 - xy)^2} dxdy \),其中
\( D = [0,1] \times [0,1] \)。

\end{document}
