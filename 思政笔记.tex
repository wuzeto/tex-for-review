\documentclass[UTF8]{ctexart}
\usepackage{amsmath}
\usepackage{amssymb}
\usepackage{graphicx}
\usepackage{booktabs}
\usepackage{amsmath}
\usepackage{amssymb}
\usepackage{graphicx}
\usepackage{booktabs}
\usepackage{tikz}
\usepackage{geometry}
\usetikzlibrary{shapes, arrows.meta, positioning, chains, decorations.pathreplacing, calligraphy}
\usepackage[dvipsnames, svgnames, x11names]{xcolor}
\usepackage{tcolorbox}
\usepackage[normalem]{ulem}

\title{思政笔记}
\author{}
\date{}
% 设置页面边距
\geometry{a4paper, margin=2cm}
\begin{document}
	\maketitle
	
	\section{考试分值与题型}
	\section*{\colorbox{green!50}{第一部分~~~\hfill 马克思主义基本原理}}
	根据考试大纲规定,本课程分值为 22 分。但从 2025 年真题看,分值为 24 分,题型分布如下。
	\begin{table}[h]
		\centering
		\begin{tabular}{ccccc}
			\toprule
			题型 & 数量 & 每题分值 & 总分值 & 真题序号 \\
			\midrule
			单项选择题 & 4 & 1 & 4 & 1、2、3、4 \\
			多项选择题 & 5 & 2 & 10 & 17、18、19、20、21 \\
			材料分析题 & 1 & 10 & 10 & 34 \\
			\bottomrule
		\end{tabular}
	\end{table}
	
	\section*{\colorbox{green!50}{第二部分~~~\hfill 毛泽东思想中国特色社会主义理论体系概论}}
	
	\subsection*{考试分值与题型}
	根据考试大纲规定,本课程分值为 13 分,但从 2025 年真题看,分值为 4 分,题型分布如下。
	\begin{table}[h]
		\centering
		\begin{tabular}{lcccc}
			\toprule
			题型 & 数量 & 每题分值 & 总分值 & 真题序号 \\
			\midrule
			单项选择题 & 2 & 1 & 2 & 5、6 \\
			多项选择题 & 1 & 2 & 2 & 22 \\
			\bottomrule
		\end{tabular}
	\end{table}
	
	\section*{\colorbox{green!50}{第三部分~~~\hfill 习近平新时代中国特色社会主义思想概论}}
	
	\subsection*{考试分值与题型}
	根据考试大纲规定,本课程分值为 22 分。2025 年真题分值为 21 分,题型分布如下。
	\begin{table}[h]
		\centering
		\begin{tabular}{lcccc}
			\toprule
			题型 & 数量 & 每题分值 & 总分值 & 真题序号 \\
			\midrule
			单项选择题 & 3 & 1 & 3 & 7、8、9 \\
			多项选择题 & 4 & 2 & 8 & 23、24、25、26 \\
			材料分析题 & 1 & 10 & 10 & 35 \\
			\bottomrule
		\end{tabular}
	\end{table}
	
	\section*{\colorbox{green!50}{第四部分~~~\hfill 中国近现代史纲要}}
	
	\subsection*{考试分值与题型}
	根据考试大纲规定,本课程分值为 15 分。但从 2025 年真题看,分值为 20 分,题型分布如下。
	\begin{table}[h]
		\centering
		\begin{tabular}{lcccc}
			\toprule
			题型 & 数量 & 每题分值 & 总分值 & 真题序号 \\
			\midrule
			单项选择题 & 4 & 1 & 4 & 10、11、12、13 \\
			多项选择题 & 3 & 2 & 6 & 27、28、29 \\
			材料分析题 & 1 & 10 & 10 & 36 \\
			\bottomrule
		\end{tabular}
	\end{table}
	
	\section*{\colorbox{green!50}{第五部分~~~\hfill 思想道德与法治}}
	
	\subsection*{考试分值与题型}
	根据考试大纲规定,本课程分值为 15 分。2025 年真题分值为 15 分,题型分布如下。
	\begin{table}[h]
		\centering
		\begin{tabular}{lcccc}
			\toprule
			题型 & 数量 & 每题分值 & 总分值 & 真题序号 \\
			\midrule
			单项选择题 & 1 & 1 & 1 & 14 \\
			多项选择题 & 2 & 2 & 4 & 30、31 \\
			材料分析题 & 1 & 10 & 10 & 37 \\
			\bottomrule
		\end{tabular}
	\end{table}
	
\end{document}