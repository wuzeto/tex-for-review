\documentclass[UTF8]{ctexart}
\usepackage{amsmath}
\usepackage{amssymb}
\usepackage{xcolor}
\usepackage{enumitem}
\usepackage{multicol}

\usepackage{amsmath}
\usepackage{geometry}
\usepackage{amssymb}
\usepackage{graphicx}
\usepackage{booktabs}
\usepackage{extarrows}
\usepackage{makecell}
\usepackage{hyperref}
\usepackage{amsmath}
\usepackage{amssymb}
\usepackage{pgfplots}
\usepackage{amsthm}

\theoremstyle{remark}
\newtheorem{remark}{注}

\pgfplotsset{compat=1.18}  % 设置兼容版本(推荐)
\usepackage{tikz}
\usepackage{tikz}
\usetikzlibrary{shapes, arrows.meta, positioning, chains, decorations.pathreplacing, calligraphy}
\usepackage[dvipsnames, svgnames, x11names]{xcolor}
\usepackage{tcolorbox}
\usepackage[normalem]{ulem}
% 设置页面边距
\geometry{a3paper, landscape, margin=0.5cm}

\title{数学例题集}
\author{}
\date{}

\begin{document}
	\maketitle
	\begin{multicols}{2}
	\section*{16 二重积分}
	\subsection*{例16.1}
	证明:\(\sum_{n = 1}^{\infty}(u_{n + 1}-u_n)\)收敛\(\Leftrightarrow\lim_{n\rightarrow\infty}u_n\)存在.
	
	\subsection*{例16.2}
	记\(\sum_{n = 1}^{\infty}u_n\)的部分和\(S_n = u_1 + u_2+\cdots+u_n\),若\(\lim_{n\rightarrow\infty}u_n = 0\),\(\lim_{n\rightarrow\infty}S_{2n}=S\)(或\(\lim_{n\rightarrow\infty}S_{2n + 1}=S\)),证明:\(\sum_{n = 1}^{\infty}u_n=S\).
	
	\subsection*{例16.3}
	判别级数\(\sum_{n = 0}^{\infty}\frac{1}{n!}\)的敛散性.
	
	\subsection*{例16.4}
	以下结论,正确的是(~~~~).
	\begin{enumerate}[label=(\Alph*)]
		\item \(\sum_{n = 1}^{\infty}\ln\left(1+\frac{1}{n}\right)\)收敛,\(\sum_{n = 1}^{\infty}\frac{1}{n}\)收敛
		\item \(\sum_{n = 1}^{\infty}\ln\left(1+\frac{1}{n}\right)\)收敛,\(\sum_{n = 1}^{\infty}\frac{1}{n}\)发散
		\item \(\sum_{n = 1}^{\infty}\ln\left(1+\frac{1}{n}\right)\)发散,\(\sum_{n = 1}^{\infty}\frac{1}{n}\)收敛
		\item \(\sum_{n = 1}^{\infty}\ln\left(1+\frac{1}{n}\right)\)发散,\(\sum_{n = 1}^{\infty}\frac{1}{n}\)发散
	\end{enumerate}
	
	\subsection*{例16.5}
	设正项级数\(\sum_{n = 1}^{\infty}a_n\)收敛,证明:级数\(\sum_{n = 1}^{\infty}a_n^2\)收敛.
	
	\subsection*{例16.6}
	\begin{enumerate}
		\item 设\(\sum_{n = 1}^{\infty}u_n\)是正项级数,若\(\sum_{n = 1}^{\infty}\sqrt{u_nu_{n + 1}}\)收敛,且\(\{u_n\}\)单调减少,证明:\(\sum_{n = 1}^{\infty}u_n\)收敛;
		\item 设\(\sum_{n = 1}^{\infty}u_n\)和\(\sum_{n = 1}^{\infty}v_n\)都是正项级数,且\(\sum_{n = 1}^{\infty}u_n\)和\(\sum_{n = 1}^{\infty}v_n\)都收敛,证明:\(\sum_{n = 1}^{\infty}u_nv_n\)收敛.
	\end{enumerate}
	
	\subsection*{例16.7}
	设有方程\(x^{n}+nx - 1 = 0\),其中\(n\)为正整数。证明此方程存在唯一的正实根\(x_{n}\),并证明当\(\alpha>1\)时,级数\(\sum_{n = 1}^{\infty}x_{n}^{\alpha}\)收敛。
	
	\subsection*{例16.8}
	设\(\alpha>0\),级数\(\sum_{n = 2}^{\infty}\sin(n^{-\alpha}\ln n)\)收敛,则(~~~)。
	\begin{enumerate}[label=(\Alph*)]
		\item \(\alpha\leqslant1\)
		\item \(\alpha<1\)
		\item \(\alpha\geqslant1\)
		\item \(\alpha>1\)
	\end{enumerate}
	
	\subsection*{例16.9}
	设\(a_{n}>0\),\(p > 1\),且\(\lim_{n\rightarrow\infty}[n^{p}(e^{\frac{1}{n}} - 1)a_{n}]=1\),若级数\(\sum_{n = 1}^{\infty}a_{n}\)收敛,则\(p\)的取值范围是\_\_\_\_\_\_\_。
	
	\subsection*{例16.10}
	设数列\(\{a_{n}\}\),\(\{b_{n}\}\)满足\(0 < a_{n}<\frac{\pi}{2}\),\(0 < b_{n}<\frac{\pi}{2}\),\(\cos a_{n}-a_{n}=\cos b_{n}\),且级数\(\sum_{n = 1}^{\infty}b_{n}\)收敛。
	\begin{enumerate}
		\item 证明:\(\lim_{n\rightarrow\infty}a_{n}=0\);
		\item 证明:级数\(\sum_{n = 1}^{\infty}\frac{a_{n}}{b_{n}}\)收敛。
	\end{enumerate}
	
	\subsection*{例16.11}
	设\(a > 0\),则下列对级数\(\sum_{n = 1}^{\infty}\frac{a^{n}n!}{n^{n}}\)的敛散性说法正确的是(~~~)。
	\begin{enumerate}[label=(\Alph*)]
		\item 当\(0 < a < e\)时,原级数收敛;当\(a\geqslant e\)时,原级数发散
		\item 当\(0 < a < e\)时,原级数发散;当\(a\geqslant e\)时,原级数收敛
		\item 当\(0 < a < e\)时,原级数收敛;当\(a\geqslant e\)时,原级数收敛
		\item 当\(0 < a < e\)时,原级数发散;当\(a\geqslant e\)时,原级数发散
	\end{enumerate}
	
	\subsection*{例16.12}
	级数\(\sum_{n = 1}^{\infty}\left[e^{\frac{\sin^{2}\alpha n}{n^{2}}}+\left(\cos\frac{1}{\sqrt{n}}\right)^{n^{2}} - 1\right]\)(~~~)。
	\begin{enumerate}[label=(\Alph*)]
		\item 收敛
		\item 发散
		\item 敛散性与\(\alpha\)有关
		\item 无法判断
	\end{enumerate}
	
	\subsection*{例16.13}
	判别级数\(\sum_{n = 2}^{\infty}\frac{1}{n\ln n}\)的敛散性。
	
	\subsection*{例16.14}
	判别级数\(\sum_{n = 1}^{\infty}(-1)^{n - 1}\cdot\frac{1}{n}\)的敛散性。
	
	\subsection*{例16.15}
	判别级数\(\sum_{n = 1}^{\infty}\frac{(-1)^{n}}{\sqrt{n}-\ln n}\)的敛散性。
	
	\subsection*{例16.16}
	判别级数\(\sum_{n = 2}^{\infty}\frac{(-1)^{n}}{\sqrt{n}+(-1)^{n}}\)的敛散性。
	
	\subsection*{例16.17}
	已知级数\(\sum_{n = 1}^{\infty}(-1)^{n - 1}u_{n}\)条件收敛,\(u_{n}>0\),则级数\(\sum_{n = 1}^{\infty}(u_{2n}-2u_{2n - 1})\)(~~~~)。
	\begin{enumerate}[label=(\Alph*)]
		\item 发散
		\item 绝对收敛
		\item 条件收敛
		\item 敛散性无法判断
	\end{enumerate}
	
	\subsection*{例16.18}
	级数\(\sum_{n = 1}^{\infty}\left(\frac{1}{\sqrt{n}}-\frac{1}{\sqrt{n + 1}}\right)\sin(n + k)\)(\(k\)为常数)(~~~~)。
	\begin{enumerate}[label=(\Alph*)]
		\item 绝对收敛
		\item 条件收敛
		\item 发散
		\item 敛散性与\(k\)有关
	\end{enumerate}
	
	\subsection*{例16.19}
	若\(\sum_{n = 1}^{\infty}nu_{n}\)绝对收敛,\(\sum_{n = 1}^{\infty}\frac{v_{n}}{n}\)条件收敛,则(~~~)。
	\begin{enumerate}[label=(\Alph*)]
		\item \(\sum_{n = 1}^{\infty}u_{n}v_{n}\)条件收敛
		\item \(\sum_{n = 1}^{\infty}u_{n}v_{n}\)绝对收敛
		\item \(\sum_{n = 1}^{\infty}(u_{n}+v_{n})\)收敛
		\item \(\sum_{n = 1}^{\infty}(u_{n}+v_{n})\)发散
	\end{enumerate}
	
	\subsection*{例16.20}
	如果数项级数\(\sum_{n = 1}^{\infty}u_{n}\)收敛,则以下级数中必收敛的是(~~~)。
	\begin{enumerate}[label=(\Alph*)]
		\item \(\sum_{n = 1}^{\infty}(-1)^{n}\frac{u_{n}}{n}\)
		\item \(\sum_{n = 1}^{\infty}u_{n}^{2}\)
		\item \(\sum_{n = 1}^{\infty}(u_{2n - 1}-u_{2n})\)
		\item \(\sum_{n = 1}^{\infty}(u_{n}+u_{n + 1})\)
	\end{enumerate}
	
	\subsection*{例16.21}
	级数\(\sum_{n = 1}^{\infty}(-1)^{n + 1}\frac{\sqrt{n + 1}-\sqrt{n}}{n^{p}}\)条件收敛,则\(p\)的取值范围为\_\_\_\_。
	
	\subsection*{例16.22}
	设\(a_{n}=\sum_{k = 1}^{n}\frac{1}{k}\),则级数\(\sum_{n = 1}^{\infty}a_{n}x^{n}\)的收敛半径为\_\_\_\_。
	
	\subsection*{例16.23}
	幂级数\(\sum_{n = 1}^{\infty}\frac{(-1)^{n - 1}}{2n - 1}x^{2n}\)的收敛域为\_\_\_\_。
	
	\subsection*{例16.24}
	设\(f(x)=\sum_{n = 0}^{\infty}x^{n}\),\(g(x)=\int_{0}^{x}f(t)dt\),则\(f(x)\)与\(g(x)\)的收敛域分别为(~~~)。
	\begin{enumerate}[label=(\Alph*)]
		\item \((-1,1), (-1,1)\)
		\item \((-1,1), [-1,1)\)
		\item \([-1,1), (-1,1)\)
		\item \([-1,1), [-1,1]\)
	\end{enumerate}
	
	\subsection*{例16.25}
	已知\(\sum_{n = 1}^{\infty}\frac{n!}{n^{n}}e^{-nx}\)的收敛域为\((a, +\infty)\),则\(a = \)  。
	
	\subsection*{例16.26}
	设\(\sum_{n = 1}^{\infty}a_{n}(x + 1)^{n}\)在点\(x = 1\)处条件收敛,则幂级数\(\sum_{n = 1}^{\infty}na_{n}(x - 1)^{n}\)在点\(x = 2\)处(  )。
	\begin{enumerate}[label=(\Alph*)]
		\item 绝对收敛
		\item 条件收敛
		\item 发散
		\item 敛散性不确定
	\end{enumerate}
	
	\subsection*{例16.27}
	幂级数\(\sum_{n = 2}^{\infty}\left(\frac{1}{n\ln n}+\frac{1}{2^{n}}\right)x^{n}\)的收敛域为  。
	
	\subsection*{例16.28}
	求\(\sum_{n = 1}^{\infty}\left(1+\frac{1}{2}+\cdots+\frac{1}{n}\right)x^{n}\)的和函数。
	
	\subsection*{例16.29}
	设函数\(y = f(x)\)满足\(y'' + 2y' + 5y = 0\),且\(f(0)=1\),\(f'(0)= - 1\)。
	\begin{enumerate}
		\item 求\(f(x)\)的表达式;
		\item 设\(a_{n}=\int_{n\pi}^{+\infty}f(x)dx\),求\(\sum_{n = 1}^{\infty}a_{n}\)。
	\end{enumerate}
	
	\subsection*{例16.30}
	求级数\(\sum_{n = 1}^{\infty}\frac{x^{n}}{n}\)的和函数。
	
	
	
	\subsection*{例16.31}
	求级数\(\sum_{n = 1}^{\infty}nx^{n}\)的和函数。
	
	\subsection*{例16.32}
	求幂级数\(\sum_{n = 1}^{\infty}\frac{(-1)^{n - 1}}{2n - 1}x^{2n}\)的和函数。
	
	\subsection*{例16.33}
	设数列\(\{a_n\}\)满足\(a_1 = 1\),\((n + 1)a_{n + 1}=\left(n + \frac{1}{2}\right)a_n\),证明:当\(\vert x\vert < 1\)时,幂级数\(\sum_{n = 1}^{\infty}a_nx^n\)收敛,并求其和函数。
	
	\subsection*{例16.34}
	设\(a_n = \int_{0}^{1}x^n\sqrt{1 - x^2}\,\mathrm{d}x\),\(b_n = \int_{0}^{\frac{\pi}{2}}\sin^n t\,\mathrm{d}t\),\(n = 1, 2, \cdots\),计算\(\sum_{n = 1}^{\infty}\frac{(-1)^{n - 1}a_n}{b_n}\)。
	
	\subsection*{例16.35}
	设\(f(x) = \int_{0}^{\sin x}\sin t^2\,\mathrm{d}t\),\(g(x) = \sum_{n = 0}^{\infty}\frac{x^{2n + 1}}{n! + 2}\),当\(x \to 0\)时,\(f(x)\)是\(g(x)\)的(  )。
	\begin{enumerate}[label=(\Alph*)]
		\item 高阶无穷小
		\item 低阶无穷小
		\item 等价无穷小
		\item 同阶非等价无穷小
	\end{enumerate}
	
	\subsection*{例16.36}
	函数\(f(x) = \ln(1 - x + x^2)\)关于\(x\)的幂级数展开式为\(\underline{\quad\quad}\)。
	
	\subsection*{例16.37}
	将函数\(f(x) = \arctan\frac{1 + x}{1 - x}\)展开为\(x\)的幂级数。
	
	\subsection*{例16.38}
	设\(f(x) = \begin{cases} 
		x, & 0 \leqslant x \leqslant \frac{1}{2}, \\
		2 - 2x, & \frac{1}{2} < x \leqslant 1, 
	\end{cases}\),\(S(x) = \frac{a_0}{2} + \sum_{n = 1}^{\infty}a_n\cos n\pi x\),\(-\infty < x < +\infty\),其中\(a_n = 2\int_{0}^{1}f(x)\cos n\pi x\,\mathrm{d}x\)(\(n = 0, 1, 2, \cdots\)),则\(S\left(-\frac{5}{2}\right) = \underline{\quad\quad}\)。
	
	\subsection*{例16.39}
	将函数\(f(x) = 1 - x^2\)(\(0 \leqslant x \leqslant \pi\))展开成余弦级数,并求级数\(\sum_{n = 1}^{\infty}\frac{(-1)^{n + 1}}{n^2}\)。
\end{multicols}
\end{document}