\documentclass[UTF8]{ctexart}
\usepackage{amsmath, amssymb}
\usepackage{enumitem}
\usepackage{tasks}
\usepackage{graphicx}
\usepackage[a4paper, margin=0.6in]{geometry}
\usepackage{fancyhdr}
\pagestyle{fancy}
\fancyhf{}
\rhead{2025}
\lhead{II}
\rfoot{第 \thepage\ 页}

% 设置 tasks 的选项样式
\settasks{
	label=(\Alph*),
	label-width=2em,
	item-indent=3em,
	column-sep=20pt,
	before-skip=3pt,
	after-skip=3pt
}

\begin{document}
	
	\begin{enumerate}[leftmargin=*, label=\textbf{\arabic*.}]
		\item 样本数据 $2$、$8$、$14$、$16$、$20$ 的平均数为
		\begin{tasks}(4)
			\task $8$
			\task $9$
			\task $12$
			\task $18$
		\end{tasks}
		
		\item 已知 $z = 1 + i$,则 $\dfrac{1}{\overline{z}-1} =$
		\begin{tasks}(4)
			\task $-i$
			\task $i$
			\task $-1$
			\task $1$
		\end{tasks}
		
		\item 已知集合 $A = \{-4, 0, 1, 2, 8\}$,$B = \{x \mid x^{3}=x\}$,则 $A \cap B =$
		\begin{tasks}(4)
			\task $\{0, 1, 2\}$
			\task $\{1, 2, 8\}$
			\task $\{2, 8\}$
			\task $\{0, 1\}$
		\end{tasks}
		
		\item 不等式 $\dfrac{x - 4}{x - 1} \geq 2$ 的解集是
		\begin{tasks}(4)
			\task $\{x \mid -2 \leq x \leq 1\}$
			\task $\{x \mid x \leq -2\}$
			\task $\{x \mid -2 \leq x < 1\}$
			\task $\{x \mid x > 1\}$
		\end{tasks}
		
		\item 在 $\triangle ABC$ 中,$BC = 2$,$AC = 1+\sqrt{3}$,$AB=\sqrt{6}$,则角 $A =$
		\begin{tasks}(4)
			\task $45^{\circ}$
			\task $60^{\circ}$
			\task $120^{\circ}$
			\task $135^{\circ}$
		\end{tasks}
		
		\item 设抛物线 $C: y^{2}=2px$ ($p>0$) 的焦点为 $F$,点 $A$ 在 $C$ 上,过 $A$ 作 $C$ 的准线的垂线,垂足为 $B$。若直线 $BF$ 的方程为 $y = -2x + 2$,则 $|AF| =$
		\begin{tasks}(4)
			\task $3$
			\task $4$
			\task $5$
			\task $6$
		\end{tasks}
		
		\item 记 $S_n$ 为等差数列 $\{a_n\}$ 的前 $n$ 项和。若 $S_3 = 6$,$S_5 = -5$,则 $S_6 =$
		\begin{tasks}(4)
			\task $-20$
			\task $-15$
			\task $-10$
			\task $-5$
		\end{tasks}
		
		\item 已知 $0 < \alpha < \pi$,$\cos\frac{\alpha}{2} = \dfrac{\sqrt{5}}{5}$,则 $\sin\left(\alpha - \dfrac{\pi}{4}\right) =$
		\begin{tasks}(4)
			\task $\dfrac{\sqrt{2}}{10}$
			\task $\dfrac{\sqrt{2}}{5}$
			\task $\dfrac{3\sqrt{2}}{10}$
			\task $\dfrac{7\sqrt{2}}{10}$
		\end{tasks}

		\item 记 $S_n$ 为等比数列 $\{a_n\}$ 的前 $n$ 项和,$q$ 为 $\{a_n\}$ 的公比,$q > 0$。若 $S_3 = 7$,$a_3 = 1$,则
		\begin{tasks}(4)
			\task $q = \frac{1}{2}$
			\task $a_5 = \frac{1}{q}$
			\task $S_5 = 8$
			\task $a_n + S_n = 8$
		\end{tasks}
		
		\item 已知 $f(x)$ 是定义在 $\mathbb{R}$ 上的奇函数,且当 $x > 0$ 时,$f(x) = (x^2 - 3)e^x + 2$,则
		\begin{tasks}(4)
			\task $f(0) = 0$
			\task 当 $x < 0$ 时,$f(x) = -(x^2 - 3)e^{-x} - 2$
			\task $f(x) \geq 2$ 当且仅当 $x \geq \sqrt{3}$
			\task $x = -1$ 是 $f(x)$ 的极大值点
		\end{tasks}
		
		\item 双曲线 $C: \frac{x^2}{a^2} - \frac{y^2}{b^2} = 1 (a > 0, b > 0)$ 的左、右焦点分别是 $F_1$,$F_2$,左、右顶点分别为 $A_1$,$A_2$,以 $F_1F_2$ 为直径的圆与 $C$ 的一条渐近线交于 $M$,$N$ 两点,且 $\angle NA_1M = \frac{5\pi}{6}$,则
		\begin{tasks}(4)
			\task $\angle A_1MA_2 = \frac{\pi}{6}$
			\task $|MA_1| = 2|MA_2|$
			\task $C$ 的离心率为 $\sqrt{3}$
			\task 当 $a = \sqrt{2}$ 时,四边形 $NA_1MA_2$ 的面积为 $8\sqrt{3}$
		\end{tasks}
		
		\item 已知平面向量 $\boldsymbol{a} = (x, 1)$,$\boldsymbol{b} = (x - 1, 2x)$。若 $\boldsymbol{a} \perp (\boldsymbol{a} - \boldsymbol{b})$,则 $|\boldsymbol{a}| =$ \_\_\_\_\_\_
		
		\item 若 $x = 2$ 是函数 $f(x) = (x - 1)(x - 2)(x - a)$ 的极值点,则 $f(10) =$ \_\_\_\_\_\_
		
		\item 一个底面半径为 $4\,\mathrm{cm}$,高为 $9\,\mathrm{cm}$ 的封闭圆柱形容器(容器壁厚度忽略不计)内有两个半径相等的铁球,则铁球半径的最大值为 \_\_\_\_\_\_ $\mathrm{cm}$。
		
		 \item 已知函数 $ f(x) = \cos(2x + \varphi) $($ 0 \leq \varphi < \pi $),$ f(0) = \frac{1}{2} $。
		\begin{enumerate}[label=(\roman*)]
			\item 求 $ \varphi $;
			\item 设函数 $ g(x) = f(x) + f\left(x - \frac{\pi}{6}\right) $,求 $ g(x) $ 的值域和单调区间。
		\end{enumerate}
		
		\item 已知椭圆 $ C: \frac{x^2}{a^2} + \frac{y^2}{b^2} = 1 $($ a > b > 0 $)的离心率为 $ \frac{\sqrt{2}}{2} $,长轴长为 4。
		\begin{enumerate}[label=(\roman*)]
			\item 求 $ C $ 的方程;
			\item 过点 $ (0, -2) $ 的直线 $ l $ 与 $ C $ 交于 $ A $、$ B $ 两点,$ O $ 为坐标原点,若 $ \triangle OAB $ 的面积为 $ \sqrt{2} $,求 $ |AB| $。
		\end{enumerate}
		
		\item 如图,在四边形 $ ABCD $ 中,$ AB \parallel CD $,$ \angle DAB = 90^\circ $,$ F $ 为 $ CD $ 的中点,点 $ E $ 在 $ AB $ 上,$ EF \parallel AD $,$ AB = 3AD $,$ CD = 2AD $。将四边形 $ EFD'A' $ 翻折至四边形 $ EFD'A' $,使得面 $ EFD'A' $ 与面 $ EFDA $ 所成的二面角为 $ 60^\circ $。
		\begin{figure}[h]
			\centering
			\includegraphics[width=0.3\textwidth]{1807.png} % 替换为实际图片路径
		\end{figure}
		\begin{enumerate}[label=(\roman*)]
			\item 证明:$ AB \parallel $ 平面 $ CD'F $;
			\item 求面 $ BCD' $ 与面 $ EF'D'A' $ 所成的二面角的正弦值。
		\end{enumerate}
		
		\item 已知函数 $ f(x) = \ln(1 + x) - x + \frac{1}{2}x^2 - kx^3 $,其中 $ 0 < k < \frac{1}{3} $。
		\begin{enumerate}[label=(\roman*)]
			\item 证明:$ f(x) $ 在区间 $ (0, +\infty) $ 存在唯一的极值点和唯一的零点;
			\item 设 $ x_1 $,$ x_2 $ 分别为 $ f(x) $ 在区间 $ (0, +\infty) $ 的极值点和零点。
			\begin{enumerate}[label=(\alph*)]
				\item 设函数 $ g(t) = f(x_1 + t) - f(x_1 - t) $,证明:$ g(t) $ 在区间 $ (0, x_1) $ 单调递减;
				\item 比较 $ 2x_1 $ 与 $ x_2 $ 的大小,并证明你的结论。
			\end{enumerate}
		\end{enumerate}
		
		\item 甲、乙两人进行乒乓球练习,每赢一球胜者得 1 分,负者得 0 分。设每个球甲胜的概率为 $ p $($ \frac{1}{2} < p < 1 $),乙胜的概率为 $ q $,且 $ p + q = 1 $。各球的胜负相互独立,对任意整数 $ k \geq 2 $,记 $ P_k $ 为打完 $ k $ 个球后乙至少比甲多得 2 分的概率。
		\begin{enumerate}[label=(\roman*)]
			\item 求 $ P_3 $,$ P_4 $(用 $ p $ 表示);
			\item 若 $ \frac{P_4 - P_3}{P_4 - P_3} = 4 $,求 $ p $;
			\item 证明:对任意正整数 $ m $,$ P_{2m+1} - P_{2m+1} < P_{2m} - P_{2m+2} $。
		\end{enumerate}
	
	    \item (一卷18)设椭圆 $ C: \frac{x^2}{a^2} + \frac{y^2}{b^2} = 1 $($ a > b > 0 $),$ A $ 为椭圆的下端点,$ B $ 为椭圆的右端点,$ |AB| = \sqrt{10} $,且椭圆 $ C $ 的离心率为 $ \frac{2\sqrt{2}}{3} $。
	\begin{enumerate}[label=(\arabic*)]
		\item 求椭圆的标准方程;
		\item 设 $ P(m, n) $。
		
		
		 (i)设 $ Q $ 在射线 $ AP $ 上,$ |AP| \cdot |AQ| = 3 $,求 $ Q $ 的坐标(用 $ m, n $ 表示);
		
		 (ii)设直线 $ OQ $ 的斜率为 $ k_1 $,直线 $ OP $ 的斜率为 $ k_2 $,若 $ k_1 = 3k_2 $,$ M $ 为椭圆上一点,求 $ |PM| $ 的最大值。
		
	\end{enumerate}
	
	\item (一卷19)设函数 $ f(x) = 5\cos x - \cos 5x $。
	\begin{enumerate}[label=(\arabic*)]
		\item 求 $ f(x) $ 在 $ \left(0, \frac{\pi}{4}\right) $ 的最大值;
		\item 给定 $ \theta \in (0, \pi) $,设 $ a $ 为实数,证明:存在 $ y \in (a - \theta, a + \theta) $,使得 $ \cos y \leq \cos \theta $;
		\item 若存在 $ t $ 使得对任意 $ x $,都有 $ 5\cos x - \cos(5x + t) \leq b $,求 $ b $ 的最小值。
	\end{enumerate}
	\end{enumerate}
	
\end{document}